\documentclass[output=paper]{langscibook}
\ChapterDOI{10.5281/zenodo.5578816}

%This is where you put the authors and their affiliations
\author[Akinbiyi Akinlabi and others]{Akinbiyi Akinlabi\affiliation{Rutgers University} and Laura J. Downing\affiliation{University of Gothenburg} and James Essegbey\affiliation{University of Florida} and Laura McPherson\affiliation{Dartmouth College} and Katie Franich\affiliation{University of Delaware} and Lee Bickmore\affiliation{University at Albany, State University of New York} and Sharon Rose\affiliation{University of California, San Diego} and Michael Cahill\affiliation{SIL} and Michael Diercks\affiliation{Pomona College}}

%Insert your title here
\title{Introduction: Reflections on 50 years of ACAL}  
\abstract{}
% % % \abstract{The authors of this brief introduction and retrospective are the volume editors: )}
\lehead{Akinbiyi Akinlabi et al.}
\begin{document}
\maketitle

\noindent The papers in this volume were presented at the 50th Annual Conference on African Linguistics held at the University of British Columbia in 2019. The conference featured around 85 accepted talks and 60 posters, as well as six invited talks. It also featured three special workshops. One of the highlights was a workshop on the grammar of music, followed by an interactive drum language session, and an invited talk by Prof. Kofi Agawu, a distinguished Professor of Music. This type of interdisciplinarity has been embraced at ACAL over the years. 

Since the first ACAL at the University of Illinois Urbana-Champaign (UIUC) in 1970, ACAL has been held every year at 25 different universities throughout North America, and one time in Africa, in 1991 at the University of Nairobi. UIUC, the first host, also hosted four other times -- in 1979, 1989, 1999 and 2009. Other three or four-time hosts include the University of California Los Angeles, Indiana University, the Ohio State University, Michigan State University, Boston University, and the University of Florida. ACAL 50 was the fourth time that ACAL was held in Canada, and the first time on the west coast of Canada. 

The contributions to this proceedings volume span a range of theoretical topics as well as topics in descriptive and applied linguistics. The papers reflect the typological and genetic diversity of languages in Africa and also represent the breadth of the ACAL community, with papers from both students and more senior scholars, based in North America and beyond. They thus provide a snapshot on current research in African linguistics, from multiple perspectives. To mark the 50th anniversary of the conference, the present volume editors reminisce, in this introductory section, about their most memorable ACALs.

\section*{Akinbiyi Akinlabi (ACAL President)}

In March 1992, at the ACAL23 conference in Michigan, I put in a bid on behalf of Rutgers University to host the next ACAL at Rutgers. That bid failed, and the next conference, ACAL24, was held at the Ohio State University, in March 1993. Rutgers put in another bid at Ohio State and won.
Though ACAL25 was not my first ACAL, it was the 25th anniversary commemoration of the Annual Conference on African Linguistics, and so a few new things happened. This conference stood out for three reasons. First, it marked the first time that ACAL abstracts were actually reviewed. There was no way we could have fitted more than 150 submitted abstracts into three days. (Poster presentations were not an option at the time.) I remember making the proposal to review the abstracts to my senior and well-established Africanist colleagues in US universities and receiving flak for it. I was told that I would “destroy the nature of ACAL”!  Today, reviewing ACAL abstracts is now normal practice.
Secondly, it was the first time that the ACAL and ALTA (African Language Teachers Association) meetings took place at the same location, at the same time. One of the ACAL25 plenary speakers actually gave two talks, one to ALTA, and the other to ACAL.
Finally, it was at ACAL25 that ACAL began signing publication contracts with independent publishers. Prior to this time, the proceedings of ACAL were published on and off as supplements to SAL (Studies in African Linguistics), until Supplement 8, and by other avenues like University working papers. In 1980, from ACAL10, the proceedings were published by Foris Press under the series “Current Approaches to African Linguistics” volumes 1--7, until 1988 (ACAL 19). Then there was no regular publisher until ACAL 25 at Rutgers. In 1994, we signed a contract with the Africa World Press as ACAL publisher, for a series titled “Trends in African Linguistics”. The volume from Rutgers appeared as the first in the series. This was a major step that lasted until ACAL33, 2002.

\section*{Laura J. Downing (ACAL Vice President/President-Elect)}

I attended my first ACAL in 1986, when I was a second year graduate student at the University of Illinois at Urbana-Champaign (UIUC). ACAL was at Indiana University that year, the state next door, and a large group of us Africanists attended from UIUC. I have two strong memories from that conference. One involves my first encounter with Thilo Schadeberg (a Bantuist, now retired, who was based at Leiden University). At the mixer before the conference got started, he walked up to me and a fellow phonology student and, with no other introduction as I remember, asked us fiercely, “You don’t believe in autosegmental phonology, do you?” Of course, we did --  didn’t everyone? We were too intimidated to make any coherent response. I’m still rather amazed that Thilo and I became friends a few conferences later. (That’s another story.)

James Mzilikazi Khumalo (who went on to become the first African head of the African Languages Department at the University of the Witwatersrand) was a visitor at UIUC during the 1985--1986 academic year (it’s because of him that I became a Bantuist), and he attended this ACAL along with a friend visiting from South Africa, a Zulu poet who wasn’t really a linguist but he had brought along a paper to present about Tsotsitaal. As the poet read his paper, his lack of academic linguistic background became painfully obvious; the paper characterized speakers of Tsotsitaal as small-time hoodlums, taking the tone of a scolding politician rather than an academic. You could feel the tension in the room rise as he spoke, and the reaction to his talk in the discussion period -- and later in the business meeting --  was explosive. A group of mainly linguists stood up to denounce the paper as inappropriate, a reminder of why ACAL should have been respecting the academic boycotts on South Africa that were widely applied during the apartheid era. Even as I could see this political point, I could only feel sorry for the poet, who was treated like a pariah during the rest of the conference by many of the attendees. (Out of solidarity with James, we UIUC people stood by him, of course.) Anyone who knows me knows I am a political person who likes a good argument, and so I found these contentious and passionate people stimulating company. I was also drawn to the underlying camaraderie uniting this community, who I could see shared a vital interest in African linguistics and the wider academic and political contexts in which our work is carried out.

I didn’t give a talk at my first ACAL. My first ever conference talk was at the Boston ACAL held a couple of years later. There I first got to know Lee Bickmore, who was his usual supportive, engaged self in the discussion period for my talk, making me feel like my first conference talk had somehow gone well. I also remember that everyone from the West Coast was talking about moras at that conference, which reinforced my opinion that Africanists were more engaged in theoretical issues of the day than anyone else. I became committed to the ACAL community after these early experiences, as a place to present my own work to a sympathetic and well-informed audience and as a place to learn from others as an engaged participant, and I’ve attended as many ACALs as I could ever since.

\section*{James Essegbey (ACAL Executive Committee)}

My first ACAL was in 2001 in Berkeley. I went to present a joint paper on serializing languages with Felix Ameka. This was my first in several respects. It was my first trip to the United States from Leiden University where I had started a postdoctoral fellowship on the Trans-Atlantic Sprachbund. Since it was my first trip, I decided to go through Washington DC to play the tourist. I was able to get a close view of the White House for the first and last time because after the September 11 attacks of that year, the place became caged like a zoo. In Berkeley, I shared a room with Tom Güldemann, whom I remembered as very funny. The conference was also my first in meeting many great linguists whose work I had read as a student. Notable among them were Larry Hyman and Ian Maddieson who were also the organizers of the conference. The highlight of the conference was a special session on historical linguistics to discuss Dixon’s (1997) criticism of the classification of African languages. I remember two main things from that conference. The first was that I told everyone who would listen that I thought the conference in America was more fun than the ones in Europe because of the closing dancing party. Ian took to the dance floor with Olanike and their dance moves were a sight to behold. I danced to my heart’s content. The second thing is that Felix and I co-authored what I believe to be the most wrongly-cited paper at the conference. The reason is that when we shared our paper with colleagues, we informed them that it would appear in the proceedings of the conference. It wasn’t long before we started seeing that appear in some publications as the reference for the paper. Alas, that was not to be. Berkeley did not publish the proceedings of that conference so in the end we had to publish it elsewhere.

\section*{Laura McPherson (ACAL Executive Committee)}\largerpage

My first ACAL was in 2008 in Athens, Georgia. I was a senior in college, and not only was it my first ACAL, but also my first academic conference, and even my first time staying at a hotel by myself! The shuttle from the airport was packed with Africanists, and I remember thinking to myself that this was easily the most jovial group of professionals I had ever had the pleasure to be around. I was presenting my thesis research on Luganda verbal morphology with my advisor, Mary Paster. Perhaps for the better, I don’t remember too much about how the actual presentation went -- I’m sure I was nervous and spoke much too quickly -- but I do remember just how supportive everyone was that I met, going out of their way to make me feel welcome even though I was a lowly undergrad. I had stars in my eyes seeing all of these real people who had until then simply been names on a page as I did research for my thesis. It was really my first true experience of academia, realizing that research is produced by human beings who also like to get coffee, go out for a drink, or dance at banquets. 

In the years since then, I have been fortunate to deepen my friendships with those I met that very first time and make many new ones as well. And I’ve even been able to pay forward the amazing mentorship that Mary gave me by twice bringing my own undergraduates to present their work. One or both times, I can’t recall, we had four generations of undergraduate lineage sitting at a table: my students (Maggie Baird and Lucas James), me, Mary Paster, and her undergraduate advisor, Dave Odden. 
That’s the kind of special conference that ACAL is -- one you look forward to every year, that you can’t wait to share with your friends, colleagues, or students, where you know the talks will always be interesting, the audience friendly, and the banquets unforgettable. Here’s to 50 more years! 

\section*{Katie Franich (ACAL Executive Committee)}

My first ACAL was in 2011 at the University of Maryland. At the time, I was a Master’s student at Boston University, and had recently returned from my first research trip to Cameroon, working on tone in Medʉmba, a Grassfields Bantu language. Also on that trip was my colleague Nick Danis, who also presented his work on Medʉmba at ACAL that year. It was the first conference presentation either of us had ever done. We were both in the same session and each had a minor panic when we found out that Larry Hyman, an expert on Grassfields Bantu (and whose work we had been poring over for the last year), was going to be chairing our session. I remember shaking like a leaf throughout, and not much more. But Larry was wonderfully supportive to both of us, and has been ever since. Nick and I celebrated with the best-tasting beer I think I’ve ever had.

I’ve only missed a couple of ACAL conferences since 2011, and each one gets better and better. I feel very fortunate to be able to be a part of this organization, one that values research and community in equal measure. I look forward to many more years of sharing work with friends and colleagues from all over, and to helping my own students to become a part of the ACAL community.

\section*{Lee Bickmore (ACAL Past President)}

My first ACAL was in 1985, held at Yale University. Perhaps my most salient memory was discovering several important errors in my handout on the plane flight over. After arriving, I thus needed to find a computer, make the changes and reprint the handouts. Of course, at the time, being a very young scholar, I thought I must be the only one at this entire conference who could be so foolish as to not make sure my handout was perfect before setting off on the trip. I’ve later learned that this is, how shall I say it, not in fact an uncommon occurrence for grad students or faculty. When I actually gave my talk, I was unbelievably nervous as this was my very first conference presentation, and the whole setting of 30 faces staring at me was nothing like delivering it in the office of my MA advisor, Russ Schuh. When I’m nervous I speak even more quickly than usual, so I finished with several minutes remaining, for which I felt somewhat embarrassed. The talk, on Hausa-English code-switching, generated several questions by someone I idolized at the time, Carol Myers-Scotton. The other person who seemed to take sincere interest in what I said was recent UCLA grad John Singler. I remember him approaching me after the talk and telling me how interesting the data were and that he’d love to get periodic updates on my research. I can’t say how much that meant to me, and it still serves as an example of how I, and every seasoned Africanist, can play an important mentor role in helping and encouraging younger scholars.
 
As I reflect and look back on 35 years of attending ACAL conferences, I do so with great pride. I can’t count how many times I’ve bragged to my department colleagues what a great conference this is. I’ve told more people than I can count what a warm and supportive environment it is—a perfect place to present one’s first paper. And the banquet! A perfect mix of African food, great conversations at the table, and watching colleagues, some of whom may have had an extra glass of wine or two, enthusiastically exhibit their dancing skills. Over the years the size and scope of the conference has had ebbs and flows. I can remember in the late 80s and early 90s when 4 or 5 universities would come to the business meeting vying to hold the next conference. We would then look at the places it had been held in the past few years and then try to balance the geography, alternating between the West, Midwest, South and Northeast. As university budgets tightened and the number of Africanists nation-wide waned somewhat, there were years where it was more challenging to find a host. It’s heartening that as this volume goes to press, we seem to be re-entering an era where membership and would-be hosts are on the rise again. In closing, for me personally, as I look back at all the entire range of my conference participation, the overwhelming majority of great memories are from ACALs. It’s my hope that our conferences energize and inform future attendees as they have me, for many more years to come.

\section*{Sharon Rose (ACAL Treasurer)}

My first ACAL was, like Mike Cahill’s, ACAL 24 held during the LSA Linguistic Institute at the Ohio State University in 1993. I was a first year PhD student at McGill and was giving a presentation on templates in Ethio-Semitic. I was already pretty immersed in African linguistics, having completed an MA at the Université du Québec à Montréal (UQAM) as part of the African Linguistics Project where most of my fellow students were Africans from Burundi, Benin, Burkina Faso, Guinea, Morocco and Ethiopia (I had missed the ACAL held at UQAM in 1987). But this was my first foray into African linguistics in the US. Halfway through my talk, right as I finished a section of my handout, the fire alarm went off! We all traipsed outside into the humid Ohio summer air and waited for the all-clear. I decided to go ahead and finish my talk outside, with my patient audience sitting on the grass. Then I had to stop again while a loud plane went overhead and by that point, we were all just laughing. Still I soldiered on. I found out later that the alarm was pulled by some children of ACAL conference attendees, the parents of whom shall remain nameless! That gave me a great introduction to the ACAL experience -- supportive audiences, a relaxed atmosphere, ... but be prepared for anything!

Another memorable ACAL was the 40th one held at UIUC in 2009. One of the invited talks was held during the banquet that year, and like at many ACALs, invited speakers sometimes don’t respect time limits. The talk wore on, people got restless, and the bored band members started to chat loudly on their phones in Lingala. The talk ended, and we thought it was time for music and dancing. But no… first, there was a drama presentation by an intro Swahili class, and then… the band? No, then a fashion show! By the time the band took the stage, the audience was ready to let loose. Congolese music burst forth and the dancing that followed was fantastically energetic. A colleague, attending his first ACAL, turned to me in amazement and said “Is it always like this?” 

The next year, ACAL 41 was held jointly between Glendon College at York University and the University of Toronto. I was honored to be one of the invited speakers, and had the surreal experience of presenting my talk in the same lecture hall where I had first taken a linguistics class at U of T 25 years earlier. In the audience were not only some of my first linguistics professors, but also some of my former African classmates from UQAM, now professors themselves, too. It was a true full circle moment.

\section*{Mike Cahill (ACAL Secretary)}

My first ACAL, in 1993 at Ohio State, was less than a stellar success. I had an M.A. and had presented a few times at other gatherings, but this was my first time at a really major event. My presentation on “Diphthongization and Underspecification in Kɔnni” was all set. But the person who was supposed to pick us up from our hotel came late, so we got to ACAL late, and my carefully prepared and carefully timed presentation had to be jammed into half the allotted time! However, I got encouraging comments, including from one of those “way up there” scholars, Peter Ladefoged, who said my paper was “very interesting.” 

And that has been one of the hallmarks of ACAL for me -- interesting and encouraging. Though the papers and people can be quite rigorous, it has not been a conference where people are just waiting to pounce on you for a minor theoretical slip-up. One presentation I remember from perhaps a decade ago was from someone whose paper included reference to a 30-year-old theory, that she evidently thought was the latest stuff, hot off the press. Yet no one said “Don’t you know that you are WAY out of date?”, but kindly interacted with her data. 

Personal interactions with linguists both new and well-established, has always been a highlight. At that first ACAL, I was hoping to get some guidance on which university to apply to for my Ph.D. I was able to talk to some professors, \emph{and} just as important, to talk to some of their grad students, asking  “What’s it like to work under Doctor So-and-So?” That helped tremendously to narrow my choices to the four schools I eventually applied to. 

I save mentioning the actual linguistic content for the last, not because it’s unimportant, but because it’s so obvious. ACAL is a place for testing your ideas, for learning about linguistic topics that you may not have been exposed to before, and of course, as in this volume, a chance to get your work out to the linguistic public, even if it’s still somewhat “in progress.” This proceedings book is unique in that it collects papers from the landmark 50th ACAL conference. But it retains the flavor of ACAL with a very broad variety of interesting papers. 

\section*{Michael Diercks (ACAL \LaTeX\ Committee head)}

My first ACAL was the 2009 one mentioned by Sharon above; I can confirm, attendees did in fact let loose at the banquet, and it was indeed astonishing. It was an entertaining, warm welcome to the ACAL community. It was that same conference where I struggled to make it to one of the conference days because a marathon was being run in town and many roads were blocked off. I parked far away and dodged marathoners across the race route to get to Brent Henderson’s plenary talk. Such adventures are simply par for the course at ACAL!
 
My favorite thing about ACAL is what others have mentioned -- it is a big tent. Because it brings together a community of scholars with shared empirical interests (but perhaps divergent theoretical ones), the one thing you can always depend on finding at ACAL is interesting data. It is where I bring talks that have intriguing puzzles and perhaps even reasonable analyses, but which are nonetheless elusive to accommodate within our theoretical frameworks and therefore have trouble finding a place in other venues. It is also one of the few places I have encountered where there is no real conflict between descriptive/analytical syntax and theoretical syntax: in many places we are expected to either do theory or do language description. ACAL is one of the few places that embraces both wholeheartedly.  
 
This also means that it is a wonderful place to learn as a graduate student or a developing scholar: ACAL participants clearly view the conference as an opportunity to welcome and encourage new scholars, rather than to celebrate only the most accomplished. The most helpful, most critical, and most challenging comments I receive on my presentations are always at ACAL conferences. But it is also where people were the kindest to me when I was just finding my way, showing me how to take the next steps. I’m proud to be a part of this community, and I look forward to what is to come, the long speeches at banquets and all. 

\section*{Acknowledgments}

We would like to thank the colleagues who took the time to review papers for this volume, helping the editors to keep up standards for the contributions and meeting our deadlines in spite of their busy schedules: Akinbiyi Akinlabi, Nico Baier, Michael Boutin, Vicki Carstens, Rod Casali, Neil Coulter, Doreen Georgi, Christopher Green, Katharina Hartmann, Brent Henderson, Jason Kandybowicz, Ulrich Kleinewillinghöfer, Ettien Koffi, Sampson Korsah, Ruth Kramer, Nicolà Lampitelli, Mike Marlo, Laura McPherson, Jeff Mielke, Deborah Morton, Scott Myers, Ruth Raharimanantsoa, Bert Remijsen, Annie Rialland, Kristina Riedel, Jack Shoemaker, Khady Tamba, Harold Torrence, Jenneke van der Wal, and Peter Unseth. We would also like to thank the ACAL \LaTeX\ Committee (Michael Diercks, Hazel Mitchley, Matthew Faytak, Ken Steimel, Peter Jenks, and Michael Marlo) for their work in developing user-friendly \LaTeX\ tools, for providing help to authors and for other crucial technical assistance in producing this volume.

{\sloppy\printbibliography[heading=subbibliography,notkeyword=this]}
\end{document}
