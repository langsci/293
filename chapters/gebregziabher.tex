\documentclass[output=paper]{langscibook}
\ChapterDOI{10.5281/zenodo.5578846}
\author{Keffyalew Gebregziabher\affiliation{University of Toronto}}

\title[Clitics or agreement markers]
      {Clitics or agreement markers: A view from Tigrinya clausal possession and modal necessity}

\abstract{This paper discusses the connection between clausal possession and modal necessity in Tigrinya. One of the unique traits of the two constructions is that they involve the same verbal element \emph{ʔalləw-}, which hosts an object marker that tracks what appears to be a subject. Using a number of diagnostics, I first demonstrate that the object marker is an agreement affix and that it should be amenable to the operation Agree. Then, using several pieces of morphosyntactic evidence, I argue that the mismatch (the object marker tracking what looks like a subject) arises due to the fact that the DP the object marker references is a ``quirky'' argument is forced to remerge higher to escape an intervention effect. Finally, I present a syntactic analysis for clausal possession and modal necessity, claiming that \emph{ʔalləw-} is the spell-out of the {\scshape appl} head that relates two arguments in Tigrinya.}

% % % Keywords: agreement, clitics, quirky-subject, EPP, object marker, Tigrinya

\begin{document}
\SetupAffiliations{mark style=none}
\maketitle

\section{Introduction}
In Tigrinya, a Semitic language mainly spoken in Ethiopia and Eritrea, there is a verbal root $\sqrt{h-l-w}$ that marks clausal possession \REF{ex:Gebregziabher:ClP} and modal necessity \REF{ex:Gebregziabher:MoN}.\footnote{The verb also marks locational predication, and often is referred to as ``existential copula'' in the literature (e.g., \citealt{hetzron1972}). In this paper, I will simply gloss it as {\scshape have} to refer to a general clausal possession marker (see \citealt{myler16} for a similar usage), without judging it as `be' or `have'.} This verbal root, like any other transitive verbal root in the language, can be affixed with a bound morpheme, which is traditionally referred to as the \emph{object marker}. In both constructions, the bound morpheme, which I hereafter call the object marker (OM), tracks the same argument, i.e., \emph{Selam} (a feminine personal name).\footnote{I am using the term OM here pre-theoretically. In Tigrinya, since the bound morpheme occurs at a position where other ordinary OMs occur, as in \REF{ex:Gebregziabher:trans}, I will continue to refer to it as such (see \citealt{kramer14} for similar use).}

\ea\label{ex:Gebregziabher:clause}
\ea\label{ex:Gebregziabher:ClP} Clausal possession \\
\gll Selam sələstə kʼolʕut \textit{ʔalləw}-u-ww\textbf{a} \\
Selam three child.{\scshape pl} {\scshape have-3mpl.s-3fsg.o}\\
\glt `Selam has three children.'
\ex\label{ex:Gebregziabher:MoN}  Modal necessity \\
\gll Selam nɨ-ɡəza kɨ-t-məs's'ɨʔ \textit{ʔalləw}-{\textbf{wa}} \\
Selam to-house {\scshape fut-3fsg}-come {\scshape have-3fsg.o}\\
\glt `Selam has to come home.'
\z
\z

\noindent The two examples in \REF{ex:Gebregziabher:clause} are similar in the sense that they both involve the same verbal root \emph{ʔalləw-} (\textit{italicized}) and two arguments -- two DPs in \REF{ex:Gebregziabher:ClP} and a DP and a proposition (TP) in \REF{ex:Gebregziabher:MoN}. The two constructions, however, are unusual in one particular respect: They both have an OM ({\textbf{bold}}), which controls what appears to be the ``subject" in their respective clauses. That is, there is the same noun phrase, \emph{Selam}, which acts as a possessor in \REF{ex:Gebregziabher:ClP} and as an obligation bearer in \REF{ex:Gebregziabher:MoN}, being tracked by the same morpheme, realized as \emph{-(ww)a} (bold) in both as well as in a simple transitive verb `kiss' given below.

\ea\label{ex:Gebregziabher:trans} Simple transitive \\
\gll Yared nɨ-Selam \textit{sɨʔim}-u-ww\textbf{a} \\
Yared {\scshape acc-}Selam kiss.{\scshape pfv-3msg.s-3fsg.o}\\
\glt `Yared kissed Selam.'
\z\largerpage

\noindent Two questions arise from the data in \REF{ex:Gebregziabher:clause}: (i) What is the status of the OM? and (ii) How do we account for the correlation between the clausal possession (ClP) and modal necessity (MoN)?

\begin{sloppypar}
In the literature, the analysis of similar elements (like the Tigrinya OMs), across a number of languages has inspired a great deal of debate (see \citealt{nevins11a}  for an overview and reopening of the debate; see also \citealt{kramer14} based on Amharic). Conceptually, clitics and agreement markers appear different, but in practice, they can be difficult to tease apart, as both share many properties in common. For example, both can be defined as prosodically weak (unaccented) morphemes, characterized by different degrees of phonological ``affinity'' to their host (\citealt{stump01}). Their phonological properties, however, can be seen as quite separate from their morphosyntactic behaviour. As \citet[16]{stump01} points out, the fact that clitics exhibit an affix-like phonological dependency on a neighbouring word may make them difficult to distinguish from affixes, even though their morphosyntax, which is word-like, appears quite distinct.

From a morpho-syntactic point of view, clitics are generally assumed to be optional and freely movable (\citealt{corbett2006}), but agreement affixes are obligatory and strictly local (\citealt{preminger09}). In addition, from a theoretical stand point, agreement affixes are realizations of φ-features (exponed on a functional head), whereas clitics are pronominal copies of the arguments they cross-reference (\citealt{nevins11a}, \citealt{kramer14}). While these are important distinctions, since arguments for and against such fundamental syntactic distinctions abound, it is important to consider other robust (morpho)syntactic diagnostics to determine whether the OMs in \REF{ex:Gebregziabher:clause} belong to elements that cross-reference arguments or to elements that realize the φ-features of the arguments associated with them.
One of the most widely-adopted methodologies in response to this question is to run a number of diagnostics (\citealt{zwickyandpullum}) and check whether the predictions the diagnostics make are borne out in the data. In their seminal work, \citet{zwickyandpullum}, for instance, discuss a list of phonological, morphological, syntactic and semantic properties that separate clitics from (agreement) affixes. While some of their diagnostics have been deemed less relevant due to lack of current theoretically-grounded motivation (see \citealt{nevins11a} for discussion), for this paper, I adopt the general approach in \citet{zwickyandpullum} as well as in many others (e.g., \citealt{kramer14}, \citealt{oxford14}, \citealt{compton16}, a.o) in determining the status of the OMs in Tigrinya. 
\end{sloppypar}

The remainder of the paper is organized as follows: \sectref{sec:gebregziabher:2} presents some theoretical context for ClP and MoN as well as for agreement markers and clitics; \sectref{sec:gebregziabher:3} tests some predictions to determine whether OMs in Tigrinya ClP and MoN are better predicted under the clitic or the agreement approach; \sectref{sec:gebregziabher:4} presents some morphosyntactic evidence that comes to prove whether the syntactic argument the OM tracks is a true subject or not; \sectref{sec:gebregziabher:5} provides some preliminary analysis for the OM in ClP and MoN; finally, \sectref{sec:gebregziabher:6} presents reasons for the mismatch (the OM tracking an apparent subject), while \sectref{sec:gebregziabher:7} concludes with some remaining issues. 

\section{Theoretical context}\label{sec:gebregziabher:2}
This section provides a brief overview of ClP, MoN and clitics and agreement affixes. It also contains background information on theoretical assumptions relevant for these issues. 

\subsection{Possession and modal necessity}\largerpage[2]
In many languages, ClP is expressed with the same mechanism as MoN (\citealt{bhatt97}; \citealt{cowperandbjorkman2016}). It has been reported that this correlation is not only observed with languages that express possession with `have', as in \REF{ex:Gebregziabher:have}, but also with languages that express possession with `be' plus a preposition/oblique case, as in \REF{ex:Gebregziabher:beto}.

\ea\label{ex:Gebregziabher:have}
\ea The students have three assignments this week. \\ \citep[(1)]{cowperandbjorkman2016}
\ex The students have to do their homework now.
\z
\ex\label{ex:Gebregziabher:beto} Hindi/Urdu \citep[8]{bhatt97}\\
\gll John-ko seb khaa-naa/sirdard hai \\
John-{\scshape dat} apple eat-{\scshape ger}/headache be.{\scshape pres}\\
\glt `John has a headache/has to eat the apple.' 
\z

\begin{sloppypar}
\noindent The fact that languages allow the same morphosyntactic elements `have/be' and `have'\,+\,infinitive or `be'\,+\,oblique\,+\,infinitive to encode possession and MoN have invited different scholars to develop a proposal that establishes a link between the two. For instance, \citet{freeze92} proposes that possessives (as in, \emph{John has a book}) and existential/locative constructions (as in, \emph{There is a book on the table}) are derivationally related and that possessives are underlyingly `be' plus incorporated P-element (see also \citealt{harley1995}, \citealt{levinson2011}, a.o, for a similar proposal). Under this proposal, possessives are a kind of existential construction, argued to have the same underlying representation as in \REF{ex:Gebregziabher:existential}.
\end{sloppypar}

\ea\label{ex:Gebregziabher:existential}
\ea {\scshape be}\textsubscript{existential} [(a book) (to John)]
\ex {\scshape be}\textsubscript{existential} [(a book) (on the table)] 
\z
\z

\noindent In the same unificational spirit, \citet{bhatt97} claims that MoN/obligation and ClP are structurally linked and that the former is a kind of existential construction. \citeauthor{bhatt97} particularly considers two  options to syntactically analyze MoN: syntactic Control, as in \REF{ex:Gebregziabher:labelgotoA} and syntactic Raising, as in \REF{ex:Gebregziabher:labelgotoB}.

\ea\label{ex:Gebregziabher:labelgoto}
\ea\label{ex:Gebregziabher:labelgotoA} There is an obligation [({\scshape pro}$_i$ to eat an apple) (to John$_i$)]
\ex\label{ex:Gebregziabher:labelgotoB} There is an obligation [(John to eat an apple)] \\ \citep[(12)]{bhatt97}
\z
\z

\noindent According to \citeauthor{bhatt97}, while \REF{ex:Gebregziabher:labelgotoA} makes a direct link between the existential and MoN, he favours \REF{ex:Gebregziabher:labelgotoB} as an underlying structure because in it, he argues, the existence of the obligation is clearly asserted. In other words, the modality involved (whether deontic or epistemic) in this structure is clearly implicated. Intuitively, deontic and epistemic modality are different because they operate in different worlds (\citealt{kratzer2006}): while the former describes what the world is like according to someone's desires/the law, the latter describes what the world should be like according to someone's desires/the law and who should be responsible for bringing it into this desired/law-obeying state (\citealt{Bhatt:1997}). 

Over the years, linguists have taken a number of syntactic approaches to distinguishing homophonous root and epistemic modals (\citealt{matthewsonetal2005}; \citealt{vonFintel:2006}). Much of the motivation for this has been purely semantic: Epistemic modality involves expression of a relation between necessity or possibility and a proposition, while root modality involves a relation between an individual and a necessary or possible event (\citealt[967--968]{butler03}, \citealt[166--167]{matthewsonetal2005}, \citealt[2]{vonFintel:2006}, a.o). Building on \citet{cowper89}, \citet{cowperandbjorkman2016} develop a syntactic proposal for modal `have' and possessive `have' arguing that the former is like a part-whole sentence (e.g., \emph{this table has four legs}), except that the relation involves sets of worlds rather than individuals.

In \citet{cowper89} (see also \citealt{ritterandrosen1997}) it has been observed that the interpretation of `have' with nominal complements denoting events and states is largely determined by the complement itself. Cowper argues that `have' in English doesn't assign thematic roles and positing multiple lexical entries for possessive `have' is neither necessary nor desirable. Instead, she develops a theory of thematic underspecification which would account for the contextually-associated multiple interpretations of the argument associated with `have'. 
Similarly, in Tigrinya, we find evidence that the subject in clausal possession displays a number of thematic relations to the event or state associated with any pragmatically-available relation, as the examples in (\ref{ex:Gebregziabher:underspec}) illustrate.\largerpage

\ea\label{ex:Gebregziabher:underspec}
\ea (Agent) \\
\gll Dr. Yared zɨħaləfə-səmun sələstə mətʼbaħti nəyr-u-wwo\\
Dr. Yared last-week three operation {\scshape have.pst-3mpl-3msg.o}\\
\glt `Dr. Yared had three operations last week.'
\ex (Patient) \\
\gll ʔɨz-i-ħɨmum zɨħ aləfə-wərħi sələstə mətʼbaħ ti nəyr-u-wwo\\
{\scshape d-msg-}patient last-month three operation {\scshape have.pst-3mpl.s-3msg.o}\\
\glt `This patient had three operations last month.'
\ex (Experiencer) \\
\gll Selam mərzam ħɨmam-rɨsi ʔalləw-ø-wa  \\
Selam nasty pain-head {\scshape have-3msg.s-3fsg.o}\\
\glt `Selam has a nasty headache.' 
\ex (Believer)  \\
\gll Yared zɨsʼənħət dɨħɨrti ʔɨmnət ʔalləw-att-o \\
Yared old odd belief {\scshape have-3fsg.s-3msg.o} \\
\glt `Yared has an old odd belief.' 
\z
\z

\noindent In (\ref{ex:Gebregziabher:underspec}), we can observe that the verb \textit{ʔalləw-} admits arguments with different thematic roles (from agent to patient to experiencer). The different arguments involved are cross-referenced by the different forms of the OM attached on the verb. One immediate question the above data raises concerns the role of the OM and whether the verb \textit{ʔalləw-} contributes any meaning at all. Building on \citealt{ritterandrosen1993} and \citealt{cowper89}, \citet{ritterandrosen1997} propose that `have' doesn't have a lexical meaning of its own, rather the meaning is structurally determined (e.g., comes to mean causative or experiencer when it forms a complex predicate with another verb). Departing from previous proposals (e.g., \citealt{freeze92}), they argue that {\sc have} is the realization of two argument-structural functional heads (rather than conceived of as the incorporation of P into {\sc be}). In more recent Minimalism and Distributed Morphology work, the treatment of {\scshape have} has been further developed to involving different flavours of functional heads (from little \textit{v}, to varieties of \textit{Appl} to \textit{Voice}). \citet{myler16}, for example, has developed an analysis for the too-many meanings {\sc have} relates (e.g., experiencers, causers, etc.) in different languages (e.g., Quechua) based on the interaction of functional heads, such as \textit{Voice} and \textit{Appl} (embracing the view that {\scshape have} is semantically vacuous).\footnote{Some languages, such as English, allow expletive subjects, as in \emph{It had to have rained last night.} and \emph{There have to to be 1000 dollars in my account by noon (otherwise, I will be screwed)}.). See \sectref{sec:cliticsVsaffixes} for discussion.} 

Based on the above facts, and following the proposal laid out in \citet{cowper89} and in \citet{ritterandrosen1997}, I argue that the Tigrinya verb \emph{ʔalləw-} (and its perfective suppletive form \emph{nəyr-}) is a transitive raising predicate with two underspecified or contextually-determined arguments. 

Therefore, building on the cross-linguistic idea that {\scshape have} does not significantly contribute core meaning to possession and other related constructions (see \citealt{myler16}; \citealt{kim11} for a similar proposal), in \sectref{sec:analysis}, I develop a syntactic analysis for Tigrinya \emph{ʔalləw-} in the context of clausal possessives and MoN. But first, in what follows, I will determine the status of the OM attached to \emph{ʔalləw-} in both constructions.

 
\subsection{Clitics or Agreement affix}\label{sec:cliticsVsaffixes}\largerpage[2]
Ever since \citeauthor{zwickyandpullum}'s (\citeyear{zwickyandpullum}) seminal work, the two notions \textit{clitics} and \textit{(agreement) affixes} have been at the forefront of both typological and theoretical studies across different fields of investigation (e.g., cognition or grammar). Each phenomenon has gained widespread attention, especially in many well-studied individual languages and language families (see \citealt{uri95} on Romance; \citealt{franco00} on Balkan; \citealt{riedel09} on Bantu; see also \citealt{corbett2006} and \citealt{spencerandluis} for a typological discussion). What emerges from this vast body of work is that the phonology and morphosyntax of clitics by-and-large are different from that of agreement affixes, even though there is still controversy around what counts as a defining property of a clitic or an (agreement) affix (see \citealt{kramer14} for discussion). 

In fact, much of the descriptive literature does not make a distinction between the two, with either agreement or clitics often used as a cover term for both phenomena (\citealt{siewierska99}; \citealt{corbett2006}). \citet[225]{siewierska99} (cited in \citealt[3--4]{croft13}) states that ``most scholars working on agreement acknowledge that there is no good basis for differentiating between person agreement markers and anaphoric pronouns". There is also a bit of a warning from \citeauthor{corbett2006} (\citeyear[112]{corbett2006}) which states that ``a rigid classification into languages with agreement or with pronominal affixes would limit rather than enhance future research." 

However, if more robust approach with more theoretically-grounded diagnostics are adopted, many empirical differences between the two phenomena may emerge and the contrast can be accounted for accordingly. In line with this view, a more recent strand of work pursues this approach with an eye to analyzing object markers as (doubled) clitics (see \citealt{harizanov14}; \citealt{kramer14} and references cited therein). In this body of work, object clitics are often analyzed as the movement of a D(P) into a verbal complex from within a ``big DP'' as in  \figref{fig:Gebregziabher:CliticAgreeStructures}(a).  
By contrast, object markers are analyzed as the realization of φ-features on the verb (\citealt{chomsky2001}) as demonstrated in \figref{fig:Gebregziabher:CliticAgreeStructures}(b). 

\begin{figure}
\caption{Derivations of clitics and agreement affixes\label{fig:Gebregziabher:CliticAgreeStructures}}
\begin{subfigure}[b]{.5\linewidth}\centering
  \begin{forest}
  [\emph{v}P [D(P), name=DP] [VP [V] [$<$DP$>$\\{[φ:val]}, name=dp]]]
  \draw[->] (dp) [in=-90,out=-180,looseness=1.5] to (DP);
  \end{forest}
\caption{Clitic movement}
\end{subfigure}%
\begin{subfigure}[b]{.5\linewidth}\centering
\begin{forest}
[\emph{v}P [\emph{v}\\{[φ:{\uline{val}}]}, name=DP] [VP [V] [DP\\{[φ:val]}, name=dp]]]
\draw[dashed,->] (dp) [in=-90,out=-180,looseness=1.5] to (DP);
\end{forest}
\caption{Agreement}
\end{subfigure}
\end{figure}

Under this structural difference, (object) agreement is viewed as the valuation of φ-feature between a probe (e.g., \emph{v}) and a φ-bearing goal (e.g., object DP) via the operation Agree (\citealt{chomsky2001}).  

In contrast, clitics are viewed as D(eterminers) that co-occur with their DP associate in a ``big-DP" structure and dislocate to lean onto the hosting verb via the operation Move (\citealt{nevins11a}). 

With this brief background, in what follows, I present a list of diagnostics that differentiate between subject markers (SMs) and (object) clitics, assuming that SMs are bona-fide agreement markers (\citealt{nevins11a}; \citealt{kramer14}). A summary of comparison between the properties of subject agreement affixes and clitics is given in \tabref{tab:Gebregziabher:Table1} (see \citealt{franco00}, \citealt{kramer14}, a.o, for similar and other diagnostics). 
The status of the OMs will be revealed as a form of prediction once each diagnostic is tested in the context of Tigrinya. It will be shown that Tigrinya OMs (in ClP and MoN) pattern more like agreement affixes and that the Agree-based approach straightforwardly accounts for their unique properties.


\begin{table}
\caption{Distinguishing clitics and agreement affixes\label{tab:Gebregziabher:Table1}}
\begin{tabular}{llccc}
\lsptoprule
&Distinguishing properties & Clitics & Subject agreement &OM\\\midrule
1 & Freely select & yes & no & ? \\ 
2 & Sensitive to meaning &yes & no &? \\ 
3 & Optional & yes & no &? \\ 
4 &Tense invariant &yes & no &? \\  
5 &Multiple per host &yes &no &? \\  
6 &Dislocate &yes &no &? \\  
7 &Non-referential &yes &no &? \\  
8 &Exhibit formal similarity with D &yes &no &? \\  
9 &Involve Case-alteration &yes &no &? \\ 
\lspbottomrule 
\end{tabular}
\end{table}


\section{Testing the predictions}\label{sec:gebregziabher:3}
In the following subsections, I test the predictions each diagnostics listed above makes for Tigrinya OMs in order to determine whether each prediction is borne out or not in ClP and MoN.

\subsection{Free selection}
One of the diagnostics that has been adopted to distinguish clitics from agreement markers is free selection; according to this criterion, agreement affixes are not free to choose their host and should always attach to their host for the sake of gaining morphological support. Clitics, by contrast, can lean onto a host or stand on their own because they have a somewhat low degree of selection with regard to their host (see \citealt{zwickyandpullum}, \citealt{anderson05}, \citealt{spencerandluis}, a.o., for discussion). For instance, in English the possessive marker \emph{'s} as a clitic can freely attach to a range of categories, as in \emph{the queen'{\textbf{s}} hat, the queen of England'{\textbf{s}} hat, the queen we saw yesterday'{\textbf{s}} hat,} etc. without specifically choosing its host (see \citealt{spencerandluis} for discussion).

If this diagnostic is applied to Tigrinya OMs, we would expect the OMs to choose freely and lean onto different hosts if they are deemed to be clitics; however, this prediction is not borne out. In Tigrinya both subject and object markers occur only as affixes attaching to the verbal element (e.g., \emph{ʔalləw-}) and never freely select or independently exist within the constituents of their host (\ref{ex:Gebregziabher:depom1} and \ref{ex:Gebregziabher:depom2}).

\ea\label{ex:Gebregziabher:free} Clausal possession
\ea
\gll ʔɨt-a məmhɨr kɨltə məs'ħafti ʔalləw-u-{\textbf{wwa}} \hspace{15mm}  \\
{\scshape d-fsg} teacher two books {\scshape have-3mpl.s-3fsg.o}\\
\glt `The teacher has two books.'
\ex\label{ex:Gebregziabher:depom1}
\gll ʔɨt-a məmhɨr (*{\textbf{wwa}}) kɨltə məs'ħafti (*{\textbf{wwa}}) ʔalləw-u \\
{\scshape d-fsg} teacher {\scshape 3fsg.o} two books {\scshape 3fsg.o} {\scshape have-3mpl.s} \\
\glt Intended: `The teacher has two books.'
\z
\ex\label{ex:Gebregziabher:affix} Modal necessity
\ea 
\gll ʔɨt-om təmaharo kɨ-məsʼsʼ-u ʔəlləw-{\textbf{om}} \hspace{17mm}  \\
{\scshape d-mpl}  student.{\scshape pl} {\scshape fut}-come-{\scshape 3mpl.s} {\scshape have-3mpl.o} \\
\ex\label{ex:Gebregziabher:depom2}
\gll ʔɨt-om təmaharo (*{\textbf{om}}) kɨ-məsʼsʼ-u (*{\textbf{om}}) ʔalləw \\
{\scshape d-mpl}  student.{\scshape pl} {\scshape 3mpl.} {\scshape fut}-come-{\scshape 3mpl.s} {\scshape 3mpl.o} {\scshape have} \\
\glt Intended: `The students have to come.'
\z
\z

The examples in \REF{ex:Gebregziabher:free} and \REF{ex:Gebregziabher:affix} demonstrate that the OMs in Tigrinya MoN and ClP neither freely select nor independently lean onto the different constituents of the host in the verbal complex. They always have to remain, unlike typical clitics, attached to the main verb. Thus, by this diagnostic, Tigrinya OMs are agreement affixes.\footnote{An anonymous reviewer raises a question whether there are (morpho)phonological processes that apply across a verb-OM boundary but not across a verb-clitic boundary? If Zwicky \& Pullum are right in pointing out that agreement affixes display allomorphy and morphological irregularities, while clitics are expected to be invariant, then Tigrinya OMs that exhibit suppletion and morphological idiosyncrasy should be treated as agreement affixes. \citeauthor{zwickyandpullum}'s (\citeyear{zwickyandpullum}) criterion C states that affixes, unlike clitics, are characterized by idiosyncratic forms, irregularities and suppletion (\citeyear[505]{zwickyandpullum}). Morphophonological idiosyncrasy happens when either (i) the host is affected by the bound morphemes attached to it, or (ii) the phonological form of the bound morpheme cannot be predicted by general rules. In Tigrinya, we have cases where the host changes its form based on the affixation of the OM (\citealt{berhane91}, \citealt{buckley00}, a.o). Particularly, type A imperfective paradigms, unsuffixed verbs have a geminated medial consonant (e.g., \emph{yɨ-sə{\textbf{bb}}ɨr} `he is breaking'), whereas verbs suffixed with the OM drop the gemination (e.g., \emph{yɨ-sə{\textbf{b}}r-om} `he is breaking them') (see \citealt{lowenstamm96}, \citealt{rose97}, a.o, for similar description on other Ethio-Semitic languages). Crucially, with ClP and MoN, while the gemination idiosyncrasy may not be fully attested, the verbal root exhibits a clear case of suppletion (e.g., \emph{nəyr-} `had (to)' vs. \emph{ʔalləw-} `have (to)') and morphological idiosyncrasy with the imperfective paradigms, as it surfaces with the radical /h-/ if the OM is involved (e.g., \emph{yɨ-{\textbf{h}}all-u-wwa} `let her have them' and \emph{yɨ-{\textbf{h}}ɨllɨw-wa} `may have to ...') that would otherwise be unavailable in the perfective paradigms (as in \emph{ʔalləw-u-wwa} `she has (to)'). Thus, the gemination dropping and the recovery of the initial radical /h/ can be considered as morphophonological processes that apply across a verb-OM boundary, consistent with the agreement nature of the OMs.}

\subsection{Semantic sensitivity}
The tendency of clitics to affect meaning is often discussed in the literature. For instance, in Spanish, clitic doubling is often attributed to the semantic feature {\scshape animacy} (\citealt{jaeggli86}; \citealt{romero10}), in Romanian it has been argued that it is connected to humanness (\citealt{dobrovie90}), in Greek and other Balkan languages, to specificity, definiteness, and/or familiarity (see \citealt{kallulli08} and references cited therein for a detailed discussion). 

In Tigrinya, while SMs and OMs can co-reference their respective full DP associates, it appears that OMs are more semantically restricted than SMs. That is, an OM requires a specific or definite object associate with regular transitive verbs, but a SM does not have the same restriction.
For example, while a verb obligatorily attaches a SM (e.g., \emph{-a}) to reference any subject full DP as in \REF{ex:Gebregziabher:indefdef}, an OM only appears when the object noun phrase is definite or specific; compare \REF{ex:Gebregziabher:dom3} \& \REF{ex:Gebregziabher:dom4}. Note that Tigrinya uses the numeral \emph{hadə}/\emph{hanti}  `one' as an indefinite marker and that, with the object DP, an accusative ({\scshape acc}) case marker (\emph{n(ɨ)-}) attaches to it (see \sectref{ex:Gebregziabher:Case} for more on this). 

\ea\label{ex:Gebregziabher:indefdef}
\ea\label{ex:Gebregziabher:dom1} 
\gll {\textbf{ʔɨt-i}} kʼolʕa bɨrtʃɨkʼo səyr-{\textbf{u}} \\
{\scshape d-fsg} child glass break.{\scshape pfv-3fsg.s}\\
\glt `The child/girl broke a glass.'
\ex\label{ex:Gebregziabher:dom2}
\gll ({\textbf{ħant-i}}) kʼolʕa bɨrtʃɨkʼo səyr-{\textbf{a}} \\
{\scshape one-f} child glass break.{\scshape pfv-3fsg.s}\\
\glt `A child/girl broke a glass.'
\z
\ex\label{ex:Gebregziabher:defdef}
\ea\label{ex:Gebregziabher:dom3} 
\gll ʔɨt-a kʼolʕa {\textbf{n-ət-i}} bɨrtʃɨk̈o səyr-a-{\textbf{tto}} \\
{\scshape d-fsg} child {\scshape acc-one-msg} glass break.{\scshape pfv-3fsg.s-3msg.o} \\
\glt `The child/girl broke the glass.'
\ex\label{ex:Gebregziabher:dom4}
\gll *ʔɨt-a kʼolʕa {\textbf{n-\textcrh adə}} bɨrtʃɨkʼo səyr-a-{\textbf{tto}} \\
{\scshape d-fsg} child {\scshape acc-d-msg} glass break.{\scshape pfv-3fsg.s-3msg.o}\\
\glt Intended: `The child/girl broke a glass.'
\z
\z


In \REF{ex:Gebregziabher:indefdef}, we have a subject marker irrespective of the definiteness or specificity of the subject noun phrase (compare \REF{ex:Gebregziabher:dom2} with \REF{ex:Gebregziabher:dom2}, where we have a definite subject `the girl' marked by an independent determiner \emph{(ʔɨ)t-}, and an indefinite subject `a girl' marked by the numeral `one', respectively). In both cases, the SM is always required, even though there is no nominative case marker as {\scshape nom} is not morphologically marked in Tigrinya. Crucially, the OM only occurs with the definite full DP associate accompanied by an accusative case marker (reminiscent of the cross-linguistic differential object marking (DOM) phenomenon, \citealt{aissen03}; see also \citealt{danon2011} and \citealt{kalin16} for a recent discussion tying DOM to φ-features).
Thus, while any subject -- definite/specific \REF{ex:Gebregziabher:dom1} or indefinite/non-specific \REF{ex:Gebregziabher:dom2} -- obligatorily triggers subject marking on the verb, it is only when the object is definite/specific (and accompanied by an accusative case marker) that the verb bears an OM. See \sectref{ex:Gebregziabher:Case} for more on this.

However, the same observation does not carry over to ClP and MoN: OMs in both construction types are not entirely conditioned by the definiteness or specificity of the object noun phrase itself -- they appear insensitive to such semantic effects.
Compare \REF{ex:Gebregziabher:modnessdef} and \REF{ex:Gebregziabher:modnessindef}.  

\ea\label{ex:Gebregziabher:modnessdef}Definite 
\ea 
\gll {\textbf{ʔɨt-a}} səbyti sələstə ʔə\textcrh wat ʔalləw-u-{\textbf{wa}}  \\
{\scshape d-fsg} woman three brothers {\scshape have-3mpl.s-3fsg.o}\\
\glt `The woman has three brothers.'
\ex 
\gll {\textbf{ʔɨt-a}} səbyti kɨ-t-məs's'ɨʔ ʔalləw-{\textbf{wwa}} \\
{\scshape d-fsg} woman {\scshape fut-3fsg.s}-come.{\scshape ipfv} {\scshape have-3fsg.o}\\
\glt `The woman has to come.'
\z
\ex\label{ex:Gebregziabher:modnessindef}Indefinite 
\ea[?]{\label{ex:Gebregziabher:modnessindefa}
\gll  {\textbf{ħant-i}} səbyti sələstə ʔə\textcrh wat ʔalləw-u-{\textbf{wwa}}  \\
    one-{\scshape fsg} woman three children {\scshape have-3mpl.s-3fsg.o}\\
\glt `A woman has three brothers.'}
\ex[?]{\label{ex:Gebregziabher:modnessindefb}
\gll  {\textbf{ħant-i}} səbyti kɨ-t-məs's'ɨʔ ʔalləww-{\textbf{wa}} \\
       one-{\scshape fsg} woman {\scshape fut-3fsg.s}-come.{\scshape ipfv} {\scshape have-3fsg.o}\\
\glt `A woman has to come.'}
\z
\z

\noindent The examples in \REF{ex:Gebregziabher:modnessdef} and \REF{ex:Gebregziabher:modnessindef} show that definiteness does not have a significant role in determining the appearance of the OM in MoN and ClP (compared to the case with regular transitive clauses as in \REF{ex:Gebregziabher:defdef}). In \REF{ex:Gebregziabher:modnessdef}, a definite DP {\emph{ʔɨt-a səbyti}} `the woman' and an indefinite DP `a woman'  are tracked by the same OM {\textbf{\emph{-wwa}}}. This suggests that the full DP the OM tracks, in both MoN and ClP, can be definite or indefinite, even though the ones with indefinites are a little bit degraded for some speakers. It appears, though, with an appropriate context, particularly a partitive, the judgements substantially improve; for example, \REF{ex:Gebregziabher:modnessindefa} can be acceptable in the following context: suppose a statistician is collecting census data by counting individuals in a household; after the data collection, the statistician may report ``(out of many) one/a woman has three brothers". Similarly, \REF{ex:Gebregziabher:modnessindefb} can be fully licit under the following context: in a co-ed soccer league, if at least one woman/girl is required for a team to play in each game, a head-coach may announce ``(for us to play and not to forfeit the game tonight) a woman/girl has to come (to play)". Under such contexts, both constructions with partitive and specific readings become fully licit.\footnote{In some cliticizing languages, it is also natural to observe the clitic yielding a different interpretation when doubling an indirect object than a direct object (\citealt{bleam99}), which is not the case with Tigrinya OMs. I thank an anonymous reviewer for asking to clarify this issue.} In other words, while the OM is not generally sensitive to definiteness in MoN and ClP, for some speakers it appears that partitive and specific meanings seem to play a role in installing OMs onto the verb. I therefore take that OMs in MoN and ClP pattern with subject agreement markers because they are semantically insensitive (at least to some speakers) to definiteness/specificity.\footnote{ In fact, if we use a less specific possessee, something every woman or man might possess, as in for example, \textit{a piece of garment/scarf/a pair of pants}, the grammaticality judgement improves as well. Therefore, I take that the fact that the OMs exhibit a slight preference for having an OM with indefinite DPs is not indicative of their clitic status. Thanks to Sharon Rose for pointing out this idea to me.}

Thus, if it is true that subject markers are true φ-agreement markers because they are not conditioned by any semantic effects of their full DP associates (\citealt{preminger09}, \citealt{nevins11a}, \citealt{kramer14}, a.o), then OMs in Tigrinya MoN and ClP should be treated as true agreement markers that pattern like subject agreement markers in this respect.

\subsection{Optionality}
Optionality is widely discussed to distinguish agreement markers from clitics. It has been assumed that clitics are optional, whereas agreement affixes are obligatory (see \citealt{corbett2006}, \citealt{nevins11a}, \citealt{baker12}, a.o).
In Tigrinya, SMs are consistently obligatory (regardless of whether they are indefinite \REF{ex:Gebregziabher:indef1} or definite \REF{ex:Gebregziabher:defsubj}), as illustrated below. 
\begin{multicols}{2}
\ea
\ea\label{ex:Gebregziabher:indef1} 
\gll \textit{ħanti} \textit{kʼolʕa} bɨrtʃʼkʼo səyr*({\textbf{-a}}) \\
one-{\scshape f} child glass break.{\scshape pfv-3fsg.s}\\
\glt `A child broke a glass.'
\ex\label{ex:Gebregziabher:defsubj} 
\gll \textit{ʔɨt-a} \textit{kʼolʕa} bɨrtʃʼkʼo səyr*({\textbf{-a}}) \\
{\scshape d-fsg} child glass break.{\scshape pfv-3fsg.s}\\
\glt `The child broke a glass.'
\z
\z
\end{multicols}
\noindent On the other hand, OMs are obligatory as long as the associated object DPs are, as pointed out above, definite/specific. Compare \REF{ex:Gebregziabher:indefobj} with \REF{ex:Gebregziabher:defobj}. 


\ea\label{ex:Gebregziabher:indef2} 
\ea\label{ex:Gebregziabher:indefobj} Indefinite object \\
\gll Selam \textit{nɨ-ħadə} \textit{təmaharay} sʼərrif-a(*{\textbf{-tto}}) \\
Selam {\scshape acc-}one student insult.{\scshape pfv-3fsg.s-3msg.o}\\
\glt `Selam insulted a student.'

\ex\label{ex:Gebregziabher:defobj} Definite object \\
\gll Selam \textit{nɨ-t-i} \textit{təmaharay}/\textit{nɨʕɨʕu} sʼərrif-a*({\textbf{-tto}}) \\
Selam {\scshape acc-d-msg} student/him insult.{\scshape pfv-3fsg.s-3msg.o}\\
\glt `Selam insulted the student/him.'
\z
\z

\noindent With ClP and MoN, OMs are also obligatorily required, as the  examples in \REF{ex:Gebregziabher:OCP} and in \REF{ex:Gebregziabher:OMN} illustrate.

\ea
\ea\label{ex:Gebregziabher:OCP} Clausal possession \\
\gll ʔɨt-a-səbyti kɨltə kʼolʕut ʔalləw-u*({\textbf{-wwa}}) \\
{\scshape d-fsg}-woman two children {\scshape have-3mpl.s-3fsg.o}\\
\glt `The woman has two children.'

\ex\label{ex:Gebregziabher:OMN} Modal necessity \\
\gll ʔɨt-om təmaharo kə-sʼnɨʕ-u ʔalləw*({\textbf{-om}}) \\
{\scshape d-mpl}  student.{\scshape pl} {\scshape fut}-study-{\scshape 3mpl.s} {\scshape have-3mpl.o} \\
\glt `The students have to study.'
\z
\z

Thus, Tigrinya OMs can not be optional. They are rather obligatorily, not only with ClP and MoN, but also with simple transitive predicates as well (\ref{ex:Gebregziabher:indef2}). 

If optionality is a reliable property to tease apart canonical agreement from clitics (see also \citealt{kramer14}, \citealt{ana17}, a.o), then Tigrinya OMs (of the ClP and MoN type) should be treated as agreement markers.


\subsection{Tense variability}
Tense-variance, as a diagnostic for the agreement-clitic distinction, refers to the contextual allomorphy often observed sensitive to tense. \citet{nevins11a} proposes that clitics are tense-invariant because they are D-elements/pronouns. In other words, while agreement affixes can display allomorphy conditioned by tense, clitics (being pronouns) cannot.\footnote{However, since \citeauthor{nevins11a} only considers canonical agreement as subject agreement on T, it is unclear whether clitics are not also expected to show allomorphy with other adjacent hosting heads. As \citet[1082, fn55]{harizanov14} points out, the conclusion only seems to follow under the additional assumption that pronominal elements cannot show contextual allomorphy sensitive to the features of another adjacent head. This diagnostic, thus, predicts that if we find a φ-cross-referencing morpheme that exhibits variance, we can conclude that it is an agreement affix and occurs on T. On the other hand, if we do not find tense variance, then we cannot conclude whether the morpheme under investigation is conclusively an agreement affix or a clitic. In fact, that's exactly what we would predict  in Tigrinya, for the OM is arguably an agreement on Asp head (see \citealt{hamilton14} for a similar argument based on Mi'gmaq).}
 
Nevertheless, if we generally assume that clitics neither change the internal structure of their host nor exhibit morphological variation with a different hosting head (e.g., with Tense, Aspect and Mood (TAM)), while agreement markers may, depending on the TAM markers, then, it is possible to test the prediction on Tigrinya (see, for an application of this diagnostic, \citealt{kramer14} on Amharic, \citealt{compton16} on Inuit, a.o).

In other words, we can predict that agreement affixes might show allomorphy conditioned by TMA, but pronominal clitics need not. With this amendment in place, we do observe an asymmetry between the OM and SM in Tigrinya: while SMs exhibit allomorphy across aspectual forms, OMs seem to be generally invariant (see \citealt{baker12} and \citealt{kramer14} for similar observation on Amharic). 

In other words, we can predict that agreement affixes might show allomorphy conditioned by TMA, but clitics need not. With this amendment in place, we could test the asymmetry between the OM and the SM in Tigrinya and predict that while SMs exhibit allomorphy across aspectual forms, OMs should be generally invariant. 
In order to establish the argument, first observe that there is an aspectual opposition between perfect and imperfect aspect in Tigrinya: SMs appear as suffixes when the verb conjugates with perfect aspect, but as discontinuous morphemes (as prefixes and suffixes) when the verb conjugates with imperfective aspect.
Compare the examples given in \tabref{tab:Gebregziabher:Table2} with only 3rd person paradigms presented for ease of exposition.\footnote{The hyphen at the end of the imperfectives indicates the expected ensuing auxiliary.}

\begin{table}
\caption{Aspectual distinction of Tigrinya SMs\label{tab:Gebregziabher:Table2}}
\begin{tabular}{ *{5}{l} }
\lsptoprule
&Perfective &Gloss & Imperfective &Gloss \\\midrule
3msg & səβir-\textbf{u} &He broke &{\textbf{yɨ}}-səbbɨr- & He breaks/is breaking \\ 
3fsg & səβir-\textbf{a} &She broke & {\textbf{tɨ}}-səbbɨr- & She breaks/is breaking \\
3mpl & səβir-\textbf{om} &They broke & {\textbf{yɨ}}-səbr-{\textbf{u}}- & They break/are breaking \\ 
3fpl & səβir-\textbf{ən} &They broke & {\textbf{yɨ}}-səbr-{\textbf{a}}- & They break/are breaking \\
\lspbottomrule 
\end{tabular}
\end{table}

OMs, on the other hand, do not exhibit significant variation based on the aspectual opposition (or other TAM markers) on the hosting verb.
\tabref{tab:Gebregziabher:Table3} demonstrates the contrast between perfective and imperfective aspect with OMs. For ease of exposition, again, 3rd person masculine singular subject marker is used throughout the paradigms.

\begin{table}
\caption{Aspectual distinction of Tigrinya OMs\label{tab:Gebregziabher:Table3}}
\fittable{\begin{tabular}{ *{5}{l} }
\lsptoprule
&\ Perfective &Gloss & Imperfective &Gloss \\\midrule
3m & səβir-u-\textbf{wwo} &He broke him &yɨ-səβr-{\textbf{o}}- & He breaks/is breaking him \\ 
3f & səβir-u-\textbf{wwa} &He broke her &yɨ-səβr-{\textbf{a}}- & He breaks/is breaking her\\ 
3mpl & səβir-u-\textbf{wwom} &He broke them &yɨ-səβr-{\textbf{om}}- & He is breaking them \\ 
3fpl & səβir-u-\textbf{wwən} &He broke them &yɨ-səβr-{\textbf{ən}}- & He breaks/is breaking them \\
\lspbottomrule
\end{tabular}}
\end{table}

While the {\scshape 3msg} SM is realized as a suffix \emph{-u} in the perfective or as (a prefix) \emph{yɨ-} in the imperfective aspect, the corresponding OM is realized only as a suffix \mbox{\emph{-(yy)o}} or \emph{-(ww)o}, respectively. The allomorphy of the OM being observed is largely phonologically-conditioned -- to avoid vowel hiatus; for example, the selection of \emph{y/w} in the OMs \emph{-(y)o} and \emph{-wo} appears partly dependent on the phonological features of the SM and the OM.\footnote{While many of the formal variations appear phonologically conditioned, as in for example \mbox{\emph{-y-}} or \mbox{\emph{-w-}} is inserted to avoid vowel hiatus (\citealt{berhane91}), others are not (see \citealt{gebre13b} for discussion). For instance, when a {\scshape 3mpl} subject marker, {\emph{-om}}, is followed by a {\scshape 3msg} object marker, {\emph{-o}}, as in /{\emph{barrix-omm-o}}/ `they blessed him' in normal speech, it change from [{\emph{barrixəmmɨwwo}}] to [{\emph{barrixəmmo}}]. This can be taken as an allomorphic variation of the SM conditioned by the OM. Again, since allomorphic variation is not expected with clitics (as opposed to agreement affixes), Tigrinya OMs cannot be clitics in this respect.}

With ClP and MoN, the same conclusion obtains, as OMs in both constructions appear aspect-invariant as well. Consider the relevant third person paradigms given in \tabref{tab:Gebregziabher:Table4}. 

\begin{table}
\caption{Aspectual distinction of Tigrinya OMs in clausal possession \& modal necessity\label{tab:Gebregziabher:Table4}}
\fittable{\begin{tabular}{ *{5}{l} }
\lsptoprule
& Perfective &Gloss & Imperfective & Gloss\\\midrule
3m & ʔallə-/nəyru-\textbf{wwo} &He has/had (to) &yɨ-hɨllɨw-{\textbf{o}} & He will have (to)\\
3f & ʔallə-/nəyru-\textbf{wwa} &She has/had (to) &yɨ-hɨllɨw-{\textbf{a}} &She will have (to)\\ 
3mpl & allə-/nəyru-\textbf{wwom} &They have/had (to) &yɨ-hɨllɨw-{\textbf{om}} & They'll have (to)\\
3fpl & allə-/nəyru-\textbf{wwən} &They have/had (to) &yɨ-hɨllɨw-{\textbf{ən}} & They'll have (to)\\
\lspbottomrule
\end{tabular}}
\end{table}

Based on the paradigms in Tables \ref{tab:Gebregziabher:Table2}--\ref{tab:Gebregziabher:Table4}, one can conclude that OMs in Tigrinya are different from SMs because they are aspect-invariant. Crucially, when it comes to tense, OMs are not affected, as tense in Tigrinya is represented on the auxiliary and OMs never show up on auxiliary verbs (see \sectref{sec:Gebregziabher:dislocate} for more on this). Thus, at first look Tigrinya OMs appear to have clitic-like properties because they are aspect-invariant; on a closer look, however, the correlation -- morphological invariance with respect to tense -- need not strictly hold as Aspect in many clitic-doubling languages does not have any real effect (because pronouns normally do not necessarily co-vary with Aspect). In addition, tense-variance as a diagnostic becomes helpful only if the supposed OMs do exhibit variation for tense; however, strictly speaking, such correlation does not hold in Tigrinya, as Tense is normally indicated by auxiliary verbs that are immune to hosting OMs.

Therefore, I take it that tense-invariance does not offer any additional evidence for differentiating whether OMs in Tigrinya are agreement affixes or clitics, and thus, may not be directly relevant as a viable differentiating diagnostic for the status of clitic-/agreement distinction in Tigrinya.

\subsection{Multiple object marking}\largerpage

In a typical language with clitics, doubling more than one internal argument simultaneously is the norm rather than the exception. An illustrative example from Greek, where both the accusative Theme `the book' and the genitive Goal `John' are doubled, is given below:

\ea Greek \citep[548]{ana05}\\
\gll {\textbf{tu}} {\textbf{to}} edhosa to vivlio tu jani\\
{\scshape 3msg.gen} {\scshape 3msg.acc} gave.{\scshape 1sg} the book.{\scshape acc} the John.{\scshape gen}\\
\glt `I gave the book to John.' 
\z

\noindent In Tigrinya, however, only a single OM per clause is allowed, as the following ditransitive clauses demonstrate (see \citet[8]{baker12} for similar observation in Amharic).\footnote{Baker uses a similar observation to claim that in Amharic the OM is a canonical agreement marker. Baker argues that for the clitics view to pass, it requires to resorting to some kind of clitic cluster simplification process, in which one OM is deleted after/before another one at PF -- an added stipulation. In contrast, the agreement analysis has a very simple account of why an OM is unique: there is only one functional head F/AgrO in the functional architecture of the clause (just as there is only one T head), and it agrees only once. However, \citep[624--625]{kramer14} contends that the restriction on multiple OMs in Amharic is neutral between an agreement and a clitic doubling analysis. For example, if multiple probe or multiple Agree is invoked (\citealt{hiraiwa05}), then ``all that needs to be said is that Amharic does not have multiple Agree". In this respect, it appears that clitics and agreement affixes are essentially the same phenomenon under the operation Agree, and their difference ultimately, as \citealt{kramer14} clearly points out, boils down to the presence/absence of multiple probes: multiple probes in languages with multiple OMs and a single probe in languages with a single OM. However, this naturally raises the question of why, as \citep[626]{kramer14} cogently puts it, ``all clitic doubling languages seem to have multiple Agree/multiple probes''.} 


\ea\label{ex:Gebregziabher:double-obj} 
\ea\label{ex:Gebregziabher:double1} 
\gll Yared ʔɨt-i məs'haf \textit{nɨ-Selam} hiβ-u-\textbf{wwa*-wwo} \\
Yared {\scshape d-msg} book {\scshape dat-}Selam give.{\scshape pfv-3msg.s-3fsg.o}{\scshape -3msg.o}\\
%\glt `Yared gave the book to Selam.'
\ex\label{ex:Gebregziabher:double2} 
\gll Yared nɨ-\textit{t-i} \textit{məs'haf} nɨ-Selam hiβ-u-*\textbf{wwo}/-\textbf{wwa} \\
Yared {\scshape acc}-{\scshape d-msg} book {\scshape dat-}Selam give.{\scshape pfv-3msg.s-3msg.o}/-{\scshape 3fsg.o} \\
\glt Intended: `Yared gave the book to Selam.'
\z
\z\largerpage

\noindent In \REF{ex:Gebregziabher:double-obj}, we observe that multiple OMs are prohibited in the context of ditransitive clauses. For example, in \REF{ex:Gebregziabher:double1}, we see that more than one OM results in ungrammaticality. In addition, in \REF{ex:Gebregziabher:double2} tracking the theme (as opposed to the goal) generally renders ungrammaticality. Nevertheless, even though the OM can track different thematic arguments, there is always a single OM per clause permitted in the language. 

The same restriction is observed with MoN and ClP: the verb does not host more than a single OM (although testing multiple objects in the domain of ClP and MoN does not always appear to be viable).

\ea
\ea\label{ex:Gebregziabher:double4} ClP \\
\gll S. nɨ-wədd-a bɨzuħat məs'ħafti ʔalləw-u-wwa*{\textbf{-wwo}} \\
S. {\scshape acc-}son-her many books {\scshape have.3mpl.s-3fsg.o} {\scshape -3msg.o}\\
\glt `Selam has a lot of books for her son.'
\ex\label{ex:Gebregziabher:modagr4} MoN \\
\gll S. nɨ-wədd-a məs'ħafti kɨ-t-gəzʔ ʔalləw-wa{\textbf{*-wwo}} \\
S. {\scshape acc-}son-her books {\scshape fut-3fsg.s-}buy.{\scshape ipfv} {\scshape have-3fsg.o}/{\scshape -3msg.o} \\
\glt `Selam has to buy books for her son.'
\z
\z

\noindent Thus, while it is the norm in cliticizing languages to have more than one clitic pronoun, more than a single appearance of the OM renders ungrammaticality in Tigrinya. 
If this line of argument is on the right track, then OMs in Tigrinya are agreement markers (not clitics).

\subsection{Dislocation/movement}\label{sec:Gebregziabher:dislocate}
One of the defining properties of clitics is movement/dislocation (see \citealt{sport96}; \citealt{polettoandpollock2004}); and often depending on their position relative to the host or the site of their attachment, they are referred to as \emph{proclitics, enclitics}, or \emph{endo-clitics} (see also \citealt{spencerandluis} for a detailed discussion). 
When other material is involved with the host of the clitic, namely auxiliaries and adverbs, the clitic normally attaches to the auxiliary verb or gets intervened from the host by an adverb. For instance, in Romanian, the clitic \emph{l-} leans onto the auxiliary `have' and not the main verb `see', as shown below (\citealt{ana05}).\footnote{A reviewer mentions that \citet{kramer14} argues that low-positioned OMs are clitics in Amharic and that there is nothing inherent to clitic-hood that forces them to attach to auxiliaries. This, however, raises the question of why only the Amharic/Tigrinya OMs do not dislocate but (Romance) clitics do? Cross-linguistically, clitics are un-selective and inherently freely movable (\citealt{spencerandluis}); for example, when auxiliaries enter to the picture, the clitic normally attaches to them as opposed to the main verb (quite distinct from the OM in Tigrinya and Amharic). \citet[108]{uri95} points out that in Romance ``clitics need not be associated to a given verb and can thus climb", obtaining different orderings. While clitic- climbing and doubling may have a different treatment, still it is not clear why Tigrinya does not exhibit any empirical evidence for either phenomena if OMs are truly clitics in the data under investigation.}

\ea\label{ex:Gebregziabher:balkanhave} Romanian \citep[532]{ana05}\\
\gll {\textbf{l-am}} {\textbf{v\'azut}} pe Popescu \\
him.{\scshape acc}-have.{\scshape 1sg} seen pe Popescu\\
\glt `I have seen Popescu.' 
\z

\noindent In addition, in some Romance and Slavic languages, clitics can be separated from the main verb when adverbs and negation markers are involved (\citealt{giustiandstavrou08}), suggesting that movement/displacement is still the core feature of clitics. The following examples from Greek and Serbo-Croatian illustrate the point.

\ea\label{ex:Gebregziabher:adv} 
\ea Greek \citep[532]{giustiandstavrou08} \\
\gll To vivlio mu {\textbf{emena}} {\textbf{den}} pulithike katholu \\
the book me.{\scshape gen.str} not sold at.all\\
\glt `My book was not sold at all.'
\ex Serbo-Croatian \citep[75]{bovskovic04} \\
\gll Jovan {\textbf{mi}} {\textbf{ga}} ju\v ce dade \\
Jovan me.{\scshape dat} it.{\scshape acc} yesterday gave\\
\glt `Jovan gave it to me yesterday.'
\z
\z

\noindent However, OMs in Tigrinya are different. Recall that Tigrinya has auxiliary verbs, such as \emph{ʔɨyy-} `be (pres(ent))' and its suppletive forms \emph{nəbər-/nəyr-} `be (past)'. When the auxiliaries appear on the clause, neither can host the OM (only the lexical verb can) in a simple transitive clause.

\ea 
\gll Yared nə-t-om k'olʕu kɨ-mɨhr-om ʔɨyy-/nəyr-u(*\textbf{-wwom)} \\
Yared {\scshape acc-d-mpl} children {\scshape fut-}teach.{\scshape ipfv-3mpl.s} {\scshape aux.pres-/aux.pst-3msg.s(-3mpl.o)}\\
\glt `Yared will teach/would have taught the children.'
\z
 

\noindent Crucially, observe that the verb \emph{ʔalləw-} (with its root future form \emph{hɨllɨw-}) can occur with either of the auxiliaries (\emph{ʔɨyy-} or \emph{nəyr-}), and unsurprisingly, the OM does not ``lean on''  the accompanying auxiliary, regardless of whether the verb has different mood/aspectual forms. It must appear on the lexical verb. Compare the following examples: 


\ea\label{ex:Gebregziabher:be1} 
\gll Selam hadə-məʕalti k'olʕu kɨ-hɨllɨw-u-wwa ʔɨyy-om-/nəyr-om(*\textbf{-wwa)} \\
Selam one-day children {\scshape fut-have-3mpl.s-3fsg.o} {\scshape aux.ipfv-/aux.pf-3mpl.s(-3fsg.o)}\\
\glt `Selam will/would have children one day.'
\ex\label{ex:Gebregziabher:be2}
\gll Selam loməʕanti kɨ-t-məs's'ɨʔ ʔalləw-wa ʔɨyy-a-/nəyr-a(*\textbf{-wa)} \\
Selam today {\scshape fut-3fsg.s-}come.{\scshape ipfv} {\scshape have-3fsg.o} {\scshape aux.ipfv-/aux.pfv-3fsg.s-3fsg.o} \\
\glt `Selam will/would have to come today.'
\z

\noindent If Tigrinya OMs were clitics, they would have shown signs of displacement\slash dislocation when auxiliaries (and adverbs) accompany the hosting verb as in \REF{ex:Gebregziabher:be1} and \REF{ex:Gebregziabher:be2}. As we can observe from the above examples, OMs remain solely attached as suffixes to the main verb, a property fully consistent with the agreement approach.


\subsection{Referential-binding}

Referential-binding as a diagnostic is concerned with the question of whether OMs have a referential index or not. In many languages, an OM is associated with specificity, definiteness and animancy/humanness (see \citealt{aissen03}, \citealt{danon2006}, \citealt{kalin16}, a.o for discussion). 

For example, in Romance languages, where object marking is associated with specificity and/or animacy, it has been argued that clitics are not individual denoting pronouns, because they do not bear referential index (\citealt{sport96} \& later work). On the other hand, \citet{bakerandkramer18} argue that OMs in Amharic are clitic pronouns because (unlike true agreement markers) they are not possible with (non-referentially) quantified objects and/or with anaphoric objects.\footnote{\citet{bakerandkramer18} argue that quantified and anaphoric objects in Amharic can be doubled by OMs when the object itself is higher than the attachment site of the OM, preventing a Weak Crossover (WCO) violation. They attributed this distribution to the fact that the OM in Amharic is a clitic pronoun -- a referentially active element, which is dependent on the object -- and the ungrammaticality is ruled out by WCO effect and principle B of the binding theory. While variable order of a direct object and an indirect object is possible in Tigrinya (see \citealt{mason96}; \citealt{kifle07}; for discussion), a careful comparison between Tigrinya OMs and Amharic OMs should, unfortunately, remain open for a different paper.}


\ea\label{ex:Gebregziabher:amharic}
\gll Ləmma hullu-n-ɨmm səw/ ras-u-n ayyə-(*w) \\ 
Lemma every-{\scshape acc-foc} person/ self-his-{\scshape acc} see.{\scshape pfv-3msg.s-(3msg.o}) \\
\glt `Lemma saw everyone/himself.'
\z

\noindent However, in Tigrinya OMs are allowed with quantified and anaphoric objects in ClP and MoN, as the following examples illustrate.

\ea\label{ex:Gebregziabher:anaphor1} ClP \\
\gll Selam kullu ʕaynət/nay-baʕla məsʼħafti ʔalləw-u-*(wwa) \\ 
Selam all kind/{\scshape poss}-herself books {\scshape have.pfv-3mpl.s-(3msg.o})\\
\glt `Selam has every kind of/her own books.'\\
\vfill
\ex\label{ex:Gebregziabher:anaphor2} MoN \\
\gll Selam (nɨ)-kullu səb/nɨ-baʕla kɨ-tə-xɨbbɨr ʔalləw*(-wa)  \\ 
Selaam ({\scshape acc}-)every person/{\scshape acc}-herself {\scshape fut-3fsg.s}-respect.{\scshape ipfv} {\scshape have-(3msg.o})\\
\glt `Selam has to respect everyone/herself.'
\z


\noindent If referential-binding is a true diagnostic for separating clitics from agreement markers, then OMs in Tigrinya (ClP and MoN) should not be treated as clitics.


\subsection{Formal similarity with D}
Formal similarity between clitics and determiners in cliticizing languages, such as Romance suggests that they belong to one and the same category, namely the category D (\citealt{bleam99}, \citealt{uri95}; see also \citealt{preminger14}). In fact, it appears that the main proposal for clitics, namely ``big DP'' (e.g., \citealt{nevins11a}), largely rests on this view (\citealt{bleam99}). 
In Tigrinya, the definite marker is an independent pre-nominal determiner (\emph{ʔɨt-}) and is formally distinct from not only the third person singular masculine (\emph{-(ww)o}) and feminine (\emph{-(ww)a}) but also from the third person plural masculine (\emph{-om}) and feminine (\emph{-ən}) markers. 

Thus, the morphological overlap between determiners and OMs is unavailable, suggesting that OMs in Tigrinya are not clitics (provided that clitics are determiners). 


\subsection{Case-marking alternation}\label{ex:Gebregziabher:Case}

In many cliticizing languages, clitics co-vary with the case features of the associate (see \citealt{romero10} on Spanish, \citealt{ana17} on Greek, \citealt{kallulli08} on some Slavic languages, a.o). The following examples are from Greek and Spanish:

\ea
\ea Greek (\citealt{ana17})\\
\gll {\textbf{tu}} {\textbf{to}} edhosa to vivlio tu jani \\
{\scshape 3msg.gen} {\scshape 3msg.acc} gave.{\scshape 1sg} the book.{\scshape acc} the John.{\scshape gen}\\
\glt `I gave the book to John.' 

\ex Spanish (\citealt{romero10})\\
\gll {\textbf{se}} {\textbf{lo}} dio a Juan Maria, el libro\\
{\scshape 3msg-dat} {\scshape 3msg-acc} gave to Juan Maria-{\scshape nom}, the book\\
\glt `Maria gave to Juan the book.' 
\z
\z

\noindent As we can see from the above examples, each clitic has a designated case feature~-- {\scshape acc} and {\scshape dat/gen} in both languages.

In Tigrinya, however, the OM does not inflect for case altogether, even though the arguments the OM tracks alternate with case. Recall that Tigrinya is a {\scshape nom-acc} language with only the accusative or dative morphologically coded by the prefix \textit{nɨ/ǝ-} (phonologically-conditioned allomorphy). While the object that bears accusative/dative case normally triggers OM, that's not always the case. For example, while a Goal and a Theme argument can take the same accusative case marker, the verb only tracks the Goal argument with the OM attached to it. Consider the following examples: 


\ea\label{ex:Gebregziabher:theme}
\ea\label{ex:Gebregziabher:acc} 
\gll Yared ({\textbf{nɨ}}-t-i) məɡbi {\textbf{nə}}-t-a-kəlbi hib-u-wwa \\ 
Yared({\scshape nom}) {\scshape acc-d-msg} food {\scshape dat-d-fsg}-dog  give.{\scshape pfv-3mpl.s-3msg.o} \\
\glt `Yared gave (the) food to the dog.'
\ex\label{ex:Gebregziabher:goal1} 
\gll (??{\textbf{nɨ}}-)Selam (*{\textbf{nɨ}}-)bɨzuħat məħazut  ʔalləw-wwa \\ 
({\scshape acc}-)Selam ({\scshape acc}-)many friends  {\scshape have-3mpl.s-3fsg.o} \\
\glt Intended: `Selam has many friends.'
\ex\label{ex:Gebregziabher:goal2} 
\gll (*{\textbf{nɨ}}-)Selam ({\textbf{nɨ}}-)kullu səb kɨ-tə-xɨbbɨr ʔalləw-wa \\ 
({\scshape acc}-)Selam ({\scshape acc}-)every person {\scshape fut-3fsg.s}-respect.{\scshape ipfv} {\scshape have-3msg.o} \\
\glt Intended: `Selam has to respect everyone.'
\z
\z


\noindent In \REF{ex:Gebregziabher:acc}, observe that in a typical ditransitive predicate `give', the only argument morphologically required to be case-marked and cross-referenced by the OM \emph{-wwa} is the Goal `the dog', indicative of the facts that there is a disconnect between case and agreement (contra \citealt{chomsky2000}) and reminiscent of the fact that there is only a single argument per clause. 
With the ClP \REF{ex:Gebregziabher:goal1} and MoN \REF{ex:Gebregziabher:goal2}, however, neither the bearer of the obligation nor the possessee takes the {\scshape acc/dat} marker (\emph{nɨ/ə-}). It seems that the possessor permits it more so when it appears closer to the verb (although the construction is generally degraded) (see \sectref{ex:Gebregziabher:Why} for more on this). Crucially, the OM, whether it tracks an accusative/dative marked argument or not, does not change its form for Case.
Thus, the fact that the case feature of the OM does not alternate with the case features of the arguments associated with it is indicative of the view that the OMs in Tigrinya are different from typical clitics. 


\subsection{Interim summary}
So far we have diagnosed the properties of object markers (OMs) in Tigrinya ClP and MoN and observed that OMs pattern like canonical subject markers in some morph-syntactic respects (see \tabref{tab:Gebregziabher:Table5}): (i) they do not freely select their host, (ii) they are obligatory, (iii) they are insensitive to definiteness, (iv) they are not tense-invariant (rather aspect-invariant), (v) they only admit a single OM per clause, (vi) they are solely hosted by the main predicate, (vii) they do not involve case-alternation, and (viii) they are not formally the same with the D category.


\begin{table}
\caption{Distinguishing pronominal clitics \& agreement markers\label{tab:Gebregziabher:Table5}}
\fittable{\begin{tabular}{ ll ccc }
\lsptoprule
&Distinguishing &Pronominal &Subject  &OM (on $\emph{ʔalləw-}$)\\
& properties & clitic & agreement & \\\midrule
1 &Freely select & yes & no & no \\ 
2 &Sensitive to meaning &yes & no &?not always \\ 
3 &Optional & yes & no &no\\ 
4 &Tense invariant &yes & no &?not strictly\\  
5 &Multiple per host &yes &no &no\\  
6 &Dislocate &yes &no &no\\  
7 &Non-referential &yes &no &?no\\  
8 &Exhibit formal similarity with D &yes &no &no\\  
9 &Involve Case-alteration &yes &no &no\\  
%9 &Trigger idiosyncratic allomorphy &no &yes &yes \\ 
\lspbottomrule
\end{tabular}}
\end{table}


However, some of these diagnostics may not be as reliable to differentiate between agreement affixes and clitics, as has been pointed out in the literature (e.g., \citealt{ana17}). For example, tense-invariance does not seem to be helpful because while some cliticizing languages exhibit tense-invariance -- T-based distinction accompanied by allomorphy -- others do not (\citealt{harizanov14}). In Tigrinya, while there is no tense (in)variance per se, the aspect opposition seems to exhibit subject-object asymmetry. Similarly, referentiality does not give a clear cut distinction between the two (see \citealt{ana17} \& \citealt{sport18} for discussion). 

Therefore, if the force of the diagnostics lies in the numbers (the more agree\-ment-like properties an OM satisfies, the more difficult it is to treat like a clitic), Tigrinya OMs in these particular constructions, which satisfy more than half of the diagnostics listed above (excluding the controversial ones), should be treated like true agreement markers (amenable to the operation Agree). In \sectref{sec:analysis}, I develop an analysis which accounts for the φ-agreement status of the OMs in Tigrinya ClP and MoN. Before I present the analysis, however, in what follows, I will determine the status of the object argument itself that the OM tracks.


\section{The status of the argument the OM tracks}\label{ex:Gebregziabher:Why}\label{sec:gebregziabher:4}
In this section, I address the question of whether the possessor/MoN argument is a true subject or object. Recall that the OM tracks what appears to be a subject on the surface. While the notion ``subject'' has long been debated (at least since \citealt{keenan76}) and a comprehensive definition, which captures the whole intuition of the term, so far has not been proposed (see for discussion \citealt{mccloskey97}, \citealt{jaya2004}, a.o), I assume that subject is a placeholder for some prominent syntactic position with a typical formal marking (e.g., {\scshape nom} for {\scshape nom-acc} language) and an {\scshape epp} feature (\citealt{mccloskey97}; \citealt{chomsky2000}). Thus, by comparing objects with subjects in Tigrinya, an attempt will be made to provide some evidence in support of the view that the argument the OM tracks is a ``quirky'' subject, not a true object (or internal argument) in Tigrinya. Three pieces of evidence will be presented in support of this claim.
\subsection{Evidence 1: Word order and case} 

Recall Tigrinya is an SOV and a {\scshape nom-acc} language, and subjects, morphologically unmarked, normally occur at sentence-initial position, whereas objects, often morphologically marked for {\scshape acc} case, normally appear right before the verbal predicate. 
If the possessor of the ClP and the sole argument of the MoN are true objects, they are expected not only to appear right before the verb but also to take an accusative case marker. These predictions are not borne out.
Such arguments, unlike true objects (but consistent with typical subjects) in the language, (i) appear at sentence-initial position (\ref{ex:Gebregziabher:subjorder1} vs. \ref{ex:Gebregziabher:subjorder2}), and (ii) do not take the {\scshape acc} case marker, as the examples  in \REF{ex:Gebregziabher:subjorder3} and \REF{ex:Gebregziabher:subjorder4} illustrate.


\ea\label{ex:Gebregziabher:subjorder1}
\judgewidth{*??}
\ea[]{\label{ex:Gebregziabher:poss-acc1}  
\gll \textit{Yared} bɨzuħat məħazut ʔalləw-u-wwo\\
     Yared many friends {\scshape have-3mpl.s-3msg.o}\\}
%\glt Intended: `Yared has books.'
\ex[*??]{%
\gll bɨzuħat məħazut \textit{Yared} ʔalləw-u-wwo\\
     many friends Yared {\scshape have-3mpl.s-3msg.o}\\
\glt Intended: `Yared has many friends.'}
\z
\ex\label{ex:Gebregziabher:subjorder2}
\ea[]{\label{ex:Gebregziabher:nec-acc1}
\gll \textit{Selam} lomaʕanti kɨ-t-məs's'ɨʔ ʔalləw-wa\\
Selam today {\scshape fut-3fsg.s-}come {\scshape have-3fsg.o}\\}
%\glt Intended: `Selam has to come today.'
\ex[*]{\label{ex:Gebregziabher:nec-acc2}
\gll lomaʕanti kɨ-t-məs's'ɨʔ \textit{Selam} ʔalləw-wa\\
     today {\scshape fut-3fsg.s-}come Selam {\scshape have-3fsg.o}\\
\glt Intended: `Selam has to come today.'}
\z
\ex\label{ex:Gebregziabher:subjorder3}
\ea[]{\label{ex:Gebregziabher:poss-acc2}  
\gll (?{\textbf{nɨ}}-)\textit{Yared} bɨzuħat məħazut ʔalləw-u-wwo\\
{\scshape acc}-Yared many friends {\scshape have-3mpl.s-3msg.o}\\
\glt Intended: `Yared has many friends.'}
\ex[]{\label{ex:Gebregziabher:nec-acc3}
\gll (*{\textbf{nɨ}}-)\textit{Selam} kɨ-t-məs's'ɨʔ ʔalləw-wa\\
{\scshape acc}-Selam {\scshape fut-3fsg.s-}come {\scshape have-3fsg.o}\\
\glt Intended: `Selam has to come.'}
\z
\z

\noindent In addition, the possessee or the propositional complement do not take the {\scshape acc} case marker.


\ea\label{ex:Gebregziabher:subjorder4}
\ea[]{\label{ex:Gebregziabher:poss-acc3}  
\gll \textit{Yared} (*{\textbf{nɨ}}-)bɨzuħat məħazut ʔalləw-u-wwo\\
Yared {\scshape acc}-many friends {\scshape have-3mpl.s-3msg.o}\\
\glt Intended: `Yared has many friends.'}
\ex[]{ 
\gll \textit{Selam} (*{\textbf{nɨ}}-)kɨ-t-məs's'ɨʔ ʔalləw-wa\\
Selam {\scshape acc}-{\scshape fut-3fsg.s-}come {\scshape have-3fsg.o}\\
\glt Intended: `Selam has to come.'}
\z
\z


\noindent While the word order facts consistently show that the appearance of such arguments in the standard pre-verbal object position is illicit in both constructions, the Case-marking facts appear to show a mixed result (compare \ref{ex:Gebregziabher:subjorder3} and \ref{ex:Gebregziabher:subjorder4}). Two things deserve an explanation here. The first one concerns the presence of the accusative case marker with the possessor (which may at first sight suggest that the possessor is a true object). First notice that the construction is marginal for the speakers I have consulted, and it does not improve even when the accusative-marked possessor appears closer to the verb (\ref{ex:Gebregziabher:acc2}a,b) (see also \sectref{ex:Gebregziabher:Case} for discussion). Second, the grammatical judgement becomes even worse with independent pronouns \REF{ex:Gebregziabher:acc3}, which may again suggest that the possessor is not a true object (rather some kind of subject). Finally, accusative marking of the possessee always gives rise to ungrammaticality \REF{ex:Gebregziabher:subjorder4} (again suggesting that such arguments are not true objects, rather some kind of subjects).   


\ea\label{ex:Gebregziabher:acc2}
\judgewidth{??}
\ea[??]{\label{ex:Gebregziabher:poss-acc4}  
\gll məs'ħafti {\textbf{nɨ}Yared} ʔalləw-u-wwo\\
     books {\scshape acc}-Yared/him  {\scshape have-3msg.s-3fsg.o}\\
\glt Intended: `Yared/he has books.'}
\ex[??]{\label{ex:Gebregziabher:acc4}
\gll məs'ħafti {\textbf{nɨʕuʕu}}/{\textbf{nɨssu}} ʔalləw-u-wwo\\
     books him/he  {\scshape have-3msg.s-3fsg.o}\\
\glt Intended: `He has books.'}
\ex[*]{\label{ex:Gebregziabher:acc3}
\gll   lomaʕanti kɨ-t-məs's'ɨʔ {\textbf{nɨʕaʕa}} ʔalləw-wa\\
      today {\scshape fut-3fsg.s-}come her {\scshape have-3fsg.o}\\
\glt Intended: `She has to come today.'}
\z
\z

\noindent The second point that deserves an explanation concerns accusative-case-mark\-ing with the obligation bearer of the modal necessity argument. In order to determine the case of the matrix subject of the MoN, it has been suggested that reference needs to be made to the underlying position of the argument that undergoes raising via the chain being created (\citealt{bhatt97}). In Tigrinya, recall that non-pronoominal subjects (or external arguments) get nominative case, which is always realized by the default empty (∅), and objects (or internal arguments) get accusative case (when marked definite). Thus, the fact that the obligation bearer in \REF{ex:Gebregziabher:obl-acc} (someone implicitly implicated here), which serves as the subject of the embedded clause `to clean', cannot take an accusative pronoun \REF{ex:Gebregziabher:obl-acc} strongly suggests that the bearer of the obligation cannot be a true object in Tigrinya. In fact, the fact that the construction is licit with a bona-fide nominative pronoun strongly suggests that the obligation bearer is some kind of subject.


\ea\label{ex:Gebregziabher:obl-acc}
\gll nə-t-i ʔax'uħut (*{\textbf{nɨʕaʕa}}/{\textbf{nɨssa}}) kɨ-t-ħas'bo ʔalləw-wa\\
{\scshape acc-d-msg} utensils her/she  {\scshape fut-3fsg.s-}wash {\scshape have-3fsg.o}\\
\glt Intended: `She has to do the dishes.'
\z

\noindent Based on the above evidence, I conclude that such arguments are not true objects/internal arguments (rather some kind of `quirky' subjects) in Tigrinya.


\subsection{Evidence 2: Passivization}\label{sec:SOR}

The second piece of evidence comes from passivization. If such arguments are true objects in Tigrinya, they should be able to undergo passivization. This prediction is not borne out.
Note that Tigrinya marks passivization by prefixing {\emph{tə-}} on the verb and by introducing an optional by-phrase associated with the demoted subject (\ref{ex:Gebregziabher:passive}). 


\ea\label{ex:Gebregziabher:passive} 
\gll ʔɨt-i məs'ħaf (bɨ-Yared) {\textbf{tə}}-sərixʼ-u\\
{\scshape d-msg} book by-Yared {\scshape pass}-steal.{\scshape pfv-3msg.s}\\
\glt `The book was stolen (by Yared).'
\z


\noindent With ClP and MoN, passivization of the argument being tracked by the OM is not allowed as illustrated in \REF{ex:Gebregziabher:passive1}.\footnote{Of course, I am largely glossing over the fact that some verbs including `have' in some languages (e.g., English: \emph{*A book/a sister was had by John}) may not undergo passivization, but in some others they do (e.g., Icelandic and Quechua, see \citealt{myler16} for discussion). Even in English, passivization of `have' is widely attested than many assume (e.g., \emph{A terrible fight was had at the station} and \emph{A fierce discussion/debate was had/needs to be had to resolve the possession issue}). Nevertheless, it could very well be the case that the Tigrinya \emph{ʔalləw-} is one of those verbs that resist passivization as well. I leave further exploration of this issue open. Thanks to anonymous reviewer for asking to elaborate on this issue.}


\ea\label{ex:Gebregziabher:passive1}
\ea[*]{%
\gll məs'ħafti (bɨ-Yared) {\textbf{tə}}-nəyr-u(-wwo)\\
books by-Yared {\scshape pass-have.pst-3msg.s-3msg.o}\\
\glt Intended: `*Books were had by Yared.'}
\ex[*]{%
\gll nɨssa k'əlt'ifa kɨ-t-məs's'ɨʔ (bɨ-Selam) {\textbf{tə}}-nəyr-u(-wa)\\
she hurriedly {\scshape fut-3fsg.s-}come by-Selam {\scshape pass-have.pst-3msg.s-3msg.o}\\
\glt Intended: `She was had/made to come hurriedly by Selam.'}
\z
\z

\noindent If passivization is a true property of objects, then the arguments being tracked by the OM in ClP and MoN are not true objects. To make the argument complete (and for what it is worth), the other arguments, such as the possessee, even though they are not tracked by the OM, they do not undergo passivization as well, which is indicative of their non-object status.


\ea\label{ex:Gebregziabher:passive2}
\ea[*]{%
\gll Yared (bɨ-)məħazut {\textbf{tə}}-nəyr-u(-wwo)\\
 Yared by-friends {\scshape pass-have.pst-3msg.s-3msg.o}\\
\glt Intended: `*Yared was had by friends.'}
\ex[*]{%
\gll Selam (bɨ)-ʔax'uħut kɨ-t-ħas'ɨb {\textbf{tə}}-nəyr-u(-wa)\\
Selam by-utensils {\scshape fut-3fsg.s-}wash {\scshape pass-have.pst-3msg.s-3fsg.o}\\
\glt Intended: `Selam was had/made to do the dishes.'}
\z
\z

\subsection{Evidence 3: Subject-to-object raising predicates}

The final piece of evidence for the fact that the possessor/sole argument of the MoN is not a true object argument comes from exceptional Case-markig predicates (typically subject control, different from what I have discussed in \sectref{sec:SOR} above). Let us start with the observation that there are certain verbs, namely \emph{try, know, convince, seem}, etc., that select infinitival complements, whose subject is selected both by the main verb and the infinitival clause. Compare the following examples:


\ea\label{ex:Gebregziabher:conv}
\ea\label{ex:Gebregziabher:conv1} 
\gll Selam$_i$ kɨ-t-dɨk'k'ɨs [{\scshape pro}$_i$ fəttin-a] \\
Selam {\scshape fut-3fsg.s}-sleep.{\scshape ipfv} {} try.{\scshape pfv-3fsg.s} \\
\glt `Selam tried to sleep.'
\ex\label{ex:Gebregziabher:conv2}   
\gll Selam$_i$ kɨ-t-dɨk'k'ɨs [{\scshape pro}$_i$ ʔəʔmin-a-tta] \\
Selam {\scshape fut-3fsg.s}-sleep.{\scshape ipfv} {} persuade.{\scshape pfv-3fsg.s-3fsg.o} \\
\glt `Selam persuaded her(self) to sleep.'
\z
\z


\noindent In \REF{ex:Gebregziabher:conv1}, there are two predicates, `try' and `sleep', both sharing the same argument \emph{Selam}, which serves as the subject in both (doing the `sleeping' and the `trying').  
In \REF{ex:Gebregziabher:conv2}, by contrast, there are two predicates, `persuade' and `sleep', sharing the same argument \emph{Selam}, which serves as the subject of the matrix verb `persuade' and the object of the embedded verb `to sleep' (doing the persuasion or being persuaded and the sleeping).

The above systematic patterns make one particular prediction:  If the possessor/obligation bearer being tracked by the OM in both ClP and MoN is a true object argument, then it should be banned from undergoing subject-to-object raising under both the \emph{try}- and the \emph{convince}-type predicates (see for similar arguments \citet{landau1999} based on Hebrew and \citet{kim11} based on Korean). However, this prediction is not observed in Tigrinya as such constructions are perfectly grammatical. Compare the examples below:


\ea\label{ex:Gebregziabher:RTO1}
\ea\label{ex:Gebregziabher:sor1} 
\gll Selam bɨzuħ gənzəb kɨ-hɨllɨw-wa fəttin-a\\
Selam many money {\scshape fut-have-3fsg.o} try.{\scshape pfv-3fsg.s}\\
\glt `Selam tried to have a lot of money.'
\ex\label{ex:Gebregziabher:sor2}  
\gll Selam nabzi kɨ-t-məs's'ɨʔ kəm-zə-ʔalləw-wa fəllit'-a\\
Selam here {\scshape fut-3fsg.s-}come {\scshape comp-rel-have-3fsg.o} try.{\scshape pfv-3fsg.s}\\
\glt `Selam has to try to come here.'
\z
\ex\label{ex:Gebregziabher:RTO2}
\ea\label{ex:Gebregziabher:sor3}  
\gll Selam bɨzuħ gənzəb kɨ-hɨllɨw-wa ʔəʔmin-a-tta\\
Selam many money {\scshape fut-have-3fsg.o} persuade.{\scshape pfv-3fsg-3fsg.o} \\
\glt `Selam persuaded her(self) to have a lot of money.'
\ex\label{ex:Gebregziabher:sor4}
\gll Selam nabzi kɨ-t-məs's'ɨʔ kəm-zə-ʔalləw-wa ʔəʔmin-att-a\\
Selam here {\scshape fut-3fsg.s-}come {\scshape comp-rel-have-3fsg.o} persuade.{\scshape pfv-3fsg-3fsg.o}\\
\glt `Selam persuaded her(self) that she has to come here.'
\z
\z

\noindent The examples in \REF{ex:Gebregziabher:RTO1} and \REF{ex:Gebregziabher:RTO2} demonstrate that the possessor and sole obligation bearer argument of the MoN are not true objects. Given the verbs are ``raising predicates'', which have the property of connecting true subjects with objects via some form of a pronoun (e.g., {\scshape pro}) and that they occur at sentence-initial position, it is reasonable to conclude that these arguments in both constructions are indeed not true objects.

Based on this evidence, I conclude that both the possessor and the sole obligation bearer argument of the MoN are not true objects, but some kind of ``quirky'' subjects. In fact, if we look at the evidence from word order and case, passivization and subject-to-object raising, they are all used as diagnostics for determining subjects in many languages (see \citealt{jaya2004} on Malayalam, \citealt{holmberg05} on Finnish and Icelandic, a.o). Thus, I claim that the reason why such a mismatch occurs, i.e., the OM is tracking some kind of subject, has something to do with the nature of such arguments (as being ``quirky'') and with intervention effects, which is further discussed in \sectref{sec:gebregziabher:6}. 

Now that I have established the nature of the OM and its associate in ClP and MoN, let's develop an analysis that fully supports the view that the OM and its ``quirky" subject associate are subject to the operation Agree.  


\section{Towards an analysis}\label{sec:analysis}\label{sec:gebregziabher:5}

Assuming the standard Minimalist Program (\citealt{chomsky1995} et seq.) along with some elements of Distributed Morphology (\citealt{halleandmarantz1993} \& later work), I propose that the OM in Tigrinya is a true instance of φ-agreement realized on \emph{v} through the operation Agree and that \emph{ʔalləw-} is the spell out of the {\scshape appl} head, which introduces an argument with possession and obligation semantic features. The structure I am proposing is in \figref{fig:Gebregziabher:structure}. 

\begin{figure}
\caption{Proposed structure\label{fig:Gebregziabher:structure}}
\begin{forest}
[\emph{v}P [ApplP [DP/TP [possession/obligation, roof]] [Appl [{ʔalləw-}] ]] [\emph{v}\\{[\emph{u}φ:\uline{\scshape om}]}] ]
\end{forest}
\end{figure}

I assume that agreement happens between a probe and a goal (\citealt{chomsky2001}; see also \citealt{pesetskyandtorrego2007}, a.o, for a recent reformulation of the operation Agree), and the direction of feature valuation can be parametrized (\citealt{baker08}; see also \citealt{bjorkmanandzeijlstra2014}, for a recent discussion).

In Tigrinya, because the OM (along with the subject marker) is always attached to the main verb, object agreement involves a probe (e.g., \emph{v}) with unvalued and/or uninterpretable φ-features, and a goal (e.g., object DP) with corresponding valued and interpretable φ-features. In this case, \emph{v} (and also Asp/T) may have unvalued and/or uninterpretable φ-features indicated by \emph{unval} and \emph{uφ}, and the DP direct object (also the DP subject) may have valued and interpretable φ-features indicated by \emph{val} and \emph{i}φ. After \emph{v} establishes an Agree relation with the object noun phrase, its unvalued φ-features become valued and the case feature on the object realizes as Accusative/Dative (and the same process happens with Asp/T and the subject noun phrase).\footnote{I also assume (although nothing hinges on this) that roots combine with root-defining functional heads to form lexical items (\citealt{marantz1997})} 

\begin{figure}
\caption{\emph{v}'s downward probing and upward φ-feature valuation\label{fig:Gebregziabher:structure2}}
\begin{forest}
[\emph{v}P [VP [DP {[\emph{i}φ:\emph{\uline{val}}]},name=DP]  [V]] [{\emph{v}[\emph{u}φ: \emph{\uline{unval}}]},name=v] ]
\draw[->,overlay] ([xshift=-0.5ex]v.south) to [out=270,in=270] ([xshift=0.5ex]DP.south);
\draw[dashed, ->, overlay] ([xshift=-0.5ex]DP.south) to [out=270,in=270,looseness=1.15] ([xshift=0.5ex]v.south);
\end{forest}
\vskip1.5\baselineskip
% % % \draw[-latex](v.south)--+(0,-40pt)-|(DP);
\end{figure} 

With this background assumption, in what follows I demonstrate how both ClP and MoN are predicted by the structure given in \figref{fig:Gebregziabher:structure}.

\subsection{Clausal possession}\largerpage

With the above assumptions, I propose that a simple ClP like \textit{He has three oxen} as in \REF{ex:Gebregziabher:struct1} has the structure in \figref{fig:Gebregziabher:struct2}.\footnote{The same analysis extends to the other suppletive forms that express other TAMs.}

\ea\label{ex:Gebregziabher:struct1}
\gll nɨssa ʔəbʕur ʔalləw-u-wwa\\
she oxen {\scshape have-3mpl.s-3fsg.o}\\
\glt `She has oxen.'
\z

\begin{figure} 
\caption{Structure of \REF{ex:Gebregziabher:struct1}\label{fig:Gebregziabher:struct2}}
\begin{forest}
[AspP 
    [DP\\nɨssa]
	[Asp 
		[\emph{v}P
		[ApplP  [DP\\$<$nɨssa$>$] 
			[Appl
				[VP  [DP [ʔabʕur, roof]]
					[V [$<$$\sqrt{hlw}$$>$] 
								[$<$\emph{v}$>$]]] 
                [$<$Appl$>$\\{[ʔalləw-]}] 
            ]
        ] 
        [\emph{v}\\$\sqrt{hlw}$$+$V$+$Appl\\{[\emph{u}φ:\uline{\scshape 3fsg}]}\\{[ʔalləw-]}] ]
    [Asp\\{[{\scshape pfv/ipfv}]}\\{[{\scshape epp}]}\\{[\emph{u}φ:\uline{\scshape 3mpl}]}] 
    ]
]    
\end{forest}
\end{figure} 

The tree in \figref{fig:Gebregziabher:struct2} has to pass the following derivational steps:

\begin{enumerate}
\item First, \emph{v} probes downwards to its c-commanding domain to satisfy its [{\emph{u}}φ]-features, and finds the possessor merged in Spec, ApplP with matching features.
\item Then, the possessor raises to Spec, TP/AspP to satisfy the [{\scshape epp}] feature. 
\item Next, T/Asp probes down to its c-commanding domain to fulfil its [\emph{u}φ]-features and finds the possessee with matching features (once the intervener possessor gets out of its way).
\item Finally, Appl remerges with \emph{v} (and possibly further with T/Asp) (via M-Merger (\citealt{matushansky2006})), realizing \emph{ʔalləw-u-wwo} or one of its suppletive forms, namely \emph{nəyr-om-wwo}.
\end{enumerate}

\noindent Under this analysis, the possessor is a potential intervener that needs to be vacated in order to feed the next cycle of Agree with T/Asp (\citealt{bejarandrezac09}).  
Once the intervener is cleared out of the way, T can probe downwards to its c-commanding domain and finds the possessee `three oxen' and agrees with it, realizing a subject agreement marker. 
It is this particular mechanism that explains the ``quirky" nature of the possessor subject (see \sectref{sec:gebregziabher:6} for more on this).


\subsection{Modal necessity}\largerpage
The same mechanism works in MoN, except that the second argument of the {\scshape appl} head is a proposition or an infinitival TP (as opposed to an individual DP) and the ``quirky'' subject first raises to the specifier of Appl before it remerges in Spec, ApsP.

\ea \label{ex:Gebregziabher:MoN2}
\gll nɨssa kɨ-t-məs's'ɨʔ ʔalləw-wa\\
she {\scshape fut-3fsg.s-}come {\scshape have-3fsg.o}\\
\glt `She has to come.'
\z

\begin{figure} 
\caption{Structure of \REF{ex:Gebregziabher:MoN2}\label{fig:Gebregziabher:MoNStructure}}
\begin{forest}
[AspP [DP\\nɨssa]
      [Asp [\emph{v}P 
       [ApplP [$<$DP$>$]
        [Appl [TP [$<$$DP$$>$+kɨ-t-məs's'ɨʔ+\emph{v}$>$, roof]] [$<$Appl$>$\\{[ʔalləw-]}]]]
        [\emph{v}$+$Appl+$\sqrt{hlw}$\\{[\emph{u}φ:\uline{\scshape 3fsg}]}\\{[ʔalləw-]} ] ]
         [Asp\\{[{\scshape pfv/ipfv}]}\\{[{\scshape epp}]}] ]]
\end{forest}
\end{figure}


Similar to ClP, the above tree has to pass the following derivational steps:

\begin{enumerate}
\item First, \emph{v} probes down to its c-commanding domain to satisfy its φ-features, finds the subject of the infinitival clause TP (`she') and establishes agreement.
\item Next the subject, which is affected by the obligation, raises to Spec, TP (same as the possessor) to satisfy the [{\scshape epp}] feature. 
\item Finally, {\scshape appl} remerges with \emph{v} (and possibly with T/Asp) via M-Merger, realizing \emph{ʔalləw-} (or its suppletive forms depending on different aspectual and tense forms) with the object marker.
\end{enumerate}


\section{Why does the OM track a ``quirky'' subject?}\label{sec:gebregziabher:6}
Now recall the mismatch between the OM and the nature of the argument it tracks (i.e., the ``quirky" subject). Normally, the OM tracks true objects, but with ClP and MoN, it tracks what appears to be the subject (i.e., the possessor and the bearer of the obligation). In an attempt to explain why would this happen, I consider two possibilities: (i) the argument is underlyingly a PP with empty P (see \citealt{landau2010}; \citealt{baker12}), or (ii) the argument is an intervener with a special feature (see \citealt{rezac2008}; \citealt{bejar08}; \citealt{ana17}). While both options may have the same effect, I tentatively argue that the latter fares better than the former.

The assumption that the argument the OM tracks is underlyingly a PP is reminiscent of the ``quirky" subject phenomenon in Indo-European languages, such as Icelandic (\citealt{zaenen1985}) and in South East Asian Languages, such as Malayalam (\citealt{jaya2004}). 
A striking similarity between ``quirky'' subjects in general and ``quirky'' subjects in Tigrinya ClP and MoN is that both are not significantly different from other argument markers; i.e., they exhibit the same argument marking, word-order, and/or agreement properties. However, while ``quirky'' subjects in languages like Icelandic take a special non-nominative/dative case, which separates them from other subjects, that is not true in Tigrinya. In Tigrinya, the ``quirky'' subject is not case-marked (at least not in MoN) and tracked by an OM like other non-subject arguments are (e.g., benefactive). Instead, it is signalled by nominative case, which is morphologically unmarked (with non-pronominal DPs) in Tigrinya. 

If the assumption with the first option is correct, that the OM-associate is a PP with an empty P, the language should also permit OMs with other true PP objects and instances of predicate inversion (\citealt{denDikken2006}).
However, such predictions are not borne out in Tigrinya.
First, recall that in Tigrinya, the OM (bolded) tracks an accusative case-marked definite noun phrase (italicized), as the examples in (\ref{ex:Gebregziabher:clp-pro}) illustrate.\largerpage[-1]

\ea \label{ex:Gebregziabher:clp-pro}
   \ea \gll nɨssu \textit{nɨʕaʕa} sɨʕɨm-u{\textbf{-wwa}}\\
   he her kiss.{\scshape pfv-3msg.s-3fsg.o}\\
   \glt `He kissed her.' 
   \ex \gll Yared \textit{nɨ-Selam} sʼərif-u{\textbf{-wwa}}\\
   Yared {\scshape acc}-Selam insult.{\scshape pfv-3msg.s-3fsg.o}\\
   \glt `Yared insulted Selam.'
   \z
\z

Second, while Tigrinya bans an OM with associated non-referential PP/CP objects \REF{ex:Gebregziabher:cp1}, with PP/CP subjects, the language employs a default {\scshape 3sg.s} marking (\ref{ex:Gebregziabher:pp-1}).\footnote{An anonymous reviewer wonders about the different forms of /g-w-y/ `go' in Tigrinya and whether the geminate /yy/ stem is different from the un-geminated stem as in Amharic. Indeed, while some verbal conjugations are different (e.g., the imperfective has radical gemination in Tigrinya but not in Amharic type A verbs), generally gemination in Tigrinya, like in Amharic or other (Ethio-)Semitic languages, is morphemic and gives rise to a different type of verb paradigms (see \citealt{berhane91} for discussion). The fact that the verb `go' geminates in, for example, \emph{kəm-zɨ-xəyyɨd} but not in \emph{kəyd-a} `she has left' and exhibits different conjugations are reflections of the properties of (Ethio-)Semitic verb types in general (\citealt{hetzron1972}).} In other words, PP/CP subjects never get tracked with the OM.


\ea\label{ex:Gebregziabher:cp1} 
 \ea
 \gll \textit{wədd-u} \textit{kulɨʃaʕ} \textit{nab-bet-tɨmhɨrti} \textit{kəm-zɨ-xəyyɨd} səmiʕ-u-(*{\textbf{wwo}})\\
 son-his always to-house-education {\scshape comp-3msg.s-}go.{\scshape ipfv} hear.{\scshape pfv-3msg.s(-3msg.o)}\\
 \glt `He heard that his son always goes to school.' 
 \ex\label{ex:Gebregziabher:cp2}
 \gll (nɨssu) \textit{nab-t-i-bet-tɨmhɨrti} kəyd-u-(*{\textbf{wwo}})\\
 he to-{\scshape d-msg}-house-education go.{\scshape pfv-3msg.s(-3msg.o)}\\
 \glt `He went to the school.'
\z
\ex\label{ex:Gebregziabher:pp-1}
 \ea[]{
 \gll \textit{ɡ\textsuperscript wal-u} \textit{nab-bet-tɨmhɨrti} \textit{zəy-mɨ-xad-a} ʔannaddid-u-*({\textbf{wwo}})\\
daughter-his to-house-education {\scshape neg-nml-}go-{\scshape 3fsg.poss} annoy.{\scshape pfv-3msg.s(-3msg.o)}\\
 \glt `That his daughter didn't go to school annoyed him.'}
 \ex[]{
 \gll  \textit{tɨħti}-\textit{fɨraʃ} t'ɨrə gənzəb yɨ-ħɨbəʔəll-u-(*{\textbf{wwo}})\\
       under-mattress raw money {\scshape 3msg}-hide.{\scshape ipfv-3msg.s(-3msg.o)}\\
 \glt `Under the mattress is where you stash the cash.'}
\z
\z
 
\noindent I therefore maintain that, if Tigrinya does not allow PP object arguments to be tracked by the OM, then it would be surprising for the subject to be a PP with an empty P. In fact, in Tigrinya PPs generally function as adjuncts, and, unsurprisingly, OMs do not track them.

The second alternative, that the ``quirky'' subject is an intervener with a special feature, may explain why the mismatch arises. In a number of languages, it has been observed that the nature of ``quirky'' arguments is that they are ``in between'' cases, where certain arguments behave as if they concurrently have structural and inherent Case properties when they are compared to other arguments (\citealt{marantz91}, \citealt{mcginnis98}, \citealt{baker08}, \citealt{pesetskyandtorrego11}, \citealt{alexiadou14}, a.o). In \citet{sigurandholm08} and others, for instance, it has been argued that in Icelandic, the {\scshape dat-nom} pattern is altered due to intervention effects. In particular, intervention of the dative argument blocks agreement between the verb and the nominative argument (see also \citealt{boeckx00}, \citealt{sigurandholm08}, a.o). That is, the dative argument blocks agreement between T and the nominative argument if it remains low as in (\ref{ex:Gebregziabher:DatInt}a), but T may agree with the lower nominative argument when the dative argument vacates out of the way of the verb-probing domain as in (\ref{ex:Gebregziabher:DatInt}b) (see \citealt{sigurandholm08} for a detailed discussion).\footnote{This follows if φ-features, namely, person and number are independent probes (\citealt{bejarandrezac09}): Number agreement for example, (in the variety of Icelandic that permits) it) is possible when the dative argument moves out of the intervening position between NumP and nominative object before number probes (\citealt{holmbergandthorbjorg2003}; \citealt{sigurandholm08}).}

\ea\label{ex:Gebregziabher:DatInt}(\citealt[252]{sigurandholm08})
\ea \gll það virðst/*virðast einhverri konu myndirnar vera lj\'otar\\
{\scshape expl} seem.{\scshape 3sg/3pl} some woman.{\scshape dat} paintings.the.{\scshape nom} be ugly\\
\glt `It seems to some woman that the paintings are ugly.' 
\ex \gll Henni virðst myndirnar vera lj\'otar\\
her.{\scshape dat} seem.{\scshape 3pl} paintings.the.{\scshape nom} be ugly\\
\glt `It seems to her that the paintings are ugly.'
\z
\z

\noindent If we assume following the above discussion that ``quirky'' subjects are some kinds of experiencers or affected arguments (\citealt{mcginnis98}) and that they involve features like {\scshape inclusion} (\citealt{cowperandbjorkman2016}) in Tigrinya, the explanation may naturally fall in place. Since interveners disrupt Agree relations (\citealt{rezac2008}), ``quirky'' subjects being merged in Spec, ApplP can be considered as potential interveners for T (just as in Icelandic), which block T from establishing an Agree relationship with its potential {\scshape goal} (\figref{fig:Gebregziabher:intervene}). In many languages, the intervention is of course more visible with dative ``quirky'' arguments (see \citealt{sigurandholm08} on Icelandic; \citealt{mcginnis98} on Georgian; \citealt{rezac2008} on Basque; \citealt{ana15} on Greek), where ``quirky'' datives are assumed to have ``lexically governed'' case (\citealt{marantz91}) despite exhibiting partial agreement (see \citealt{rezac2008} and \citealt{bejarandrezac09} for proposals to derive partial vs. complete φ-features agreement from differences in the feature structures of the relevant probes).

Nevertheless, the mismatch in Tigrinya arises simply because the ``quirky'' subject is being tracked by an OM and as a potential intervener it is forced to re-merge in Spec, AspP/TP to salvage the construction from being crashed. The fact that the ``quirky'' subject involves a special feature, i.e., {\scshape inclusion}, forces it to divorce EPP from typical φ-agreement. In other words, {\scshape epp} feature checking normally happens along with standard subject-verb agreement (\citealt{chomsky2001}), but under this context, the ``quirky'' subject comes to rescue the construction from crashing by moving something without matching phi-features into Spec, AspP/TP, to satisfy the {\scshape epp} (essentially obviating the intervention); otherwise, Spec,Asp/TP would remain empty, leaving the EPP unchecked, as the structure in \figref{fig:Gebregziabher:intervene} demonstrates. 

\begin{figure} 
\caption{Structure for ClP with intervention\label{fig:Gebregziabher:intervene}}
\begin{forest}
[\emph{v}P
		[ApplP  [DP{[iφ:\uline{\scshape 3msg}]}\\Yared]
			[Appl [DP  [three oxen,roof] ]
			      [$\sqrt{hlw}$$+$v$+$Appl\\{[ʔalləw-/nəyr-]}]
            ]
        ] 
        [\emph{v} {[{\scshape incl/aff}]}\\{[\emph{u}φ:\uline{\scshape 3msg}]}]
]
\end{forest}
\end{figure} 

In \figref{fig:Gebregziabher:intervene}, when \emph{v} searches down to its c-commanding domain for a goal with a matching feature, it finds the ``quirky'' subject merged in Spec, ApplP. Since the ``quirky'' subject in Spec, ApplP has matching features (e.g., {\scshape pl}), \emph{v} agrees with it. In addition, Appl -- the head that introduces the argument being tracked by the OM -- has a feature called {[{\scshape incl}]} that requires checking by \emph{v}, before the DP argument that has matching features (e.g., goal, dative, etc.) evacuates to Spec, T/AspP, creating a workable domain for Asp/T to establish an Agree relation with the lower argument. It is through this mechanism of removing the intervention (parallel to movement of the {\scshape dat} out of the verb-probing domain we observed in Icelandic in \REF{ex:Gebregziabher:DatInt}) that the mismatch in agreement in Tigrinya could be explained. 

\begin{figure} 
\caption{Structure of ClP with intervention resolved\label{fig:Gebregziabher:interventionTree1}}
\resizebox{\textwidth}{!}{\begin{forest}
[AspP [DP\\Yared]
	[AspP
	[\emph{v}P
		[ApplP  [{$<$DP[iφ:\uline{\scshape 3msg}]$>$}]
			[Appl [DP{[iφ:\uline{\scshape 3mpl}]} [three oxen, roof]] 
                  [$<$$\sqrt{hlw}$$+$v$+$Appl$>$\\{[ʔalləw-/nəyr-]}] 
            ]
         ] 
     [{\emph{v}[{\scshape incl/aff}]}\\{[\emph{u}φ:\uline{\scshape 3msg}]}] ] 
     [Asp\\{[uφ:\uline{\scshape 3mpl}]}] ]]
\end{forest}}
\end{figure} 

One may wonder at this point whether the same intervention effect could carry over to the analysis of MoN. While the extension does not seem immediately direct as there are still some relevant morphosyntactic differences between possession and necessity, it would be a desirable consequence if the analysis could extend to MoN. Recall that in MoN the predicate may have different arguments (depending on c-selectional properties), but crucially, the surface argument involved may not be the source of the modal obligation (\citealt{bhatt97}). \citet{bhatt97} points out that the source of the obligation (whether asserted or not) is context-dependent, and argues that even in cases where the bearer of the obligation is present in the sentence, it does not have to be the subject of `have' as in \emph{Bill has to be consulted by John on every decision} (\emph{John} being the bearer of the obligation). 

\citet{cowperandbjorkman2016}, on the other hand, propose that possession and necessity establish a relation of containment or inclusion between two arguments. Building on the typological division within modal systems (e.g., \citealt{matthewsonetal2005}) and morphosyntactic featural decomposition (e.g., \citealt{harleyandritter02}), they argue that the functional head that introduces possession (\emph{v}\textsubscript{have}) carries the feature {\scshape incl(usion)}, whereas the same head that introduces MoN carries the features [{\scshape incl, epist(emic)}]; the feature [{\scshape incl}] relates two individual arguments, while the features [{\scshape incl, epist}] relate an individual with a proposition/clause. Assuming interpretable features on a head compose with each other before the result composes with the syntactic complement, they claim that \emph{v}\textsubscript{have} comes up with [{\scshape incl, epist}]  with the two features composing first. While [{\scshape epist}] provides \emph{v} with the set of best epistemically accessible worlds in the modal base, [{\scshape incl}] saturates the internal argument position the clausal complement of \emph{v} supplies.   
 
With the assumption that there is an implicit argument (whether overt or covert) in MoN, the above treatment for ClP indeed can straightforwardly be extended to the MoN domain. Assuming that \emph{Appl} relates individuals as well as sets of worlds (in the sense of \citealt{cowperandbjorkman2016}), the features [{\scshape incl, epist}] may compose first to saturate the first argument of the predicate before \emph{Appl} remerges with \emph{v} in the derivation.\largerpage

The underlying structure for MoN would therefore be as in \figref{fig:Gebregziabher:modalNess}, with [{\scshape incl, epist}] features on \emph{Appl}, and an infinitive/gerundive proposition as its syntactic complement, as illustrated in \figref{fig:Gebregziabher:modalNess}.

\begin{figure} 
\caption{Structure for MoN\label{fig:Gebregziabher:modalNess}}
\begin{forest}
[\emph{v}P
 [ApplP  [DP{[iφ:\uline{\scshape 3msg}]}\\Yared]
 [Appl [TP [$<$Yared$>$ to-come, roof] ] 
       [$\sqrt{hlw}$$+$v$+$Appl\\{[ʔalləw-/nəyr-]}]]] 
  [\emph{v}{[{\scshape incl,epist}]}\\{[\emph{u}φ:\uline{\scshape 3msg}]}] 
]
\end{forest}
\end{figure} 
  
The only difference is that the subject of the infinitive/gerund (\emph{Yared}) in Tigrinya is attracted to the specifier of ApplP through raising (as opposed to external-Merge) before it raises further to the specifier of AspP/TP to satisfy EPP. Consistent with the  analysis given for ClP, it is important to note that the DP in ClP and in MoN is not a regular DP; rather, a DP with an {\scshape incl/epst} feature which satisfies the EPP by further raising to Spec, Asp/TP as illustrated in \figref{fig:Gebregziabher:interventionTree2}. 

\begin{figure} 
\caption{Structure of MoN with intervention resolved\label{fig:Gebregziabher:interventionTree2}}
\resizebox{\textwidth}{!}{\begin{forest}
[AspP [DP{[iφ:\uline{\scshape 3msg}]}\\Yared]
	[AspP [\emph{v}P 
	     [ApplP  [$<$DP$>$]
			[Appl [TP [$<$Yared$>$ to-come,roof] ] 
			[$<$$\sqrt{hlw}$$+$v$+$Appl$>$\\{[ʔalləw-/nəyr-]}] 
			]
		 ] 
          [{\emph{v}[{\scshape incl/epist}]}\\{[\emph{u}φ:\uline{\scshape 3msg}]}] 
          ]
      [Asp\\{[uφ:\uline{\scshape 3msg}]}] 
     ]
]
\end{forest}}
\end{figure} 

Therefore, under this analysis, the mismatch arises due to the disconnect between agreement and {\scshape epp}: T/Asp needs to overcome the mismatch by attracting an argument that does not agree with it to its specifier position; since the DP that satisfies T/Asp's φ-features is not attracted to satisfy its {\scshape epp} requirement in these structures, something unusual needs to happen -- a non-agreeing DP (T/Asp typically does not agree with the object) needs to re-merge to Spec,T/AspP, clearing the intervention, so that T/Asp would agree with the sole argument (see \citealt{rezac2008}; \citealt{ana05} for similar observation based on intervention effects).

Under this analysis, modal necessity clauses in Tigrinya (perhaps similar to Hindi\slash Urdu; \citealt{bhatt97}) are raising predicates. \citet{bhatt97} offers both semantic and syntactic arguments in support of the claim that MoN constructions are indeed derived via raising (like other raising predicates; see also \citealt{landau2010}). The first argument is concerned with the fact that the subject need not be the source of the direct bearer of the obligation, as illustrated in \REF{ex:Gebregziabher:obliged}, where clearly it is not the obligation of `the door' to be fixed itself. In this case, the subject is more thematically related only to the embedded clause.

\ea\label{ex:Gebregziabher:obliged} 
\gll ʔɨt-i maʕsʼo kɨ-sʼɨɡɡən  ʔalləw-wo\\
{\scshape d-msg} door {\scshape fut-}fix.{\scshape ipfv} {\scshape have-(3msg.s)-3msg.o}\\
\glt `The door has to be fixed.'  
\z 

The second argument is syntactic and has to do with the agreement morpheme mismatch that surfaces on the verb. In Tigrinya, the OM tracks the subject, and of course the direct bearer of the obligation when implicated on the clause. Consider the following:

\ea 
\gll nɨ-ʔɨt-i maʕsʼo kɨ-sʼɨɡɡɨn-o  ʔalləw-wo\\
{\scshape d-msg} door {\scshape fut-}fix.{\scshape ipfv-3msg.o} {\scshape have-(3msg.s)-3msg.o}\\
\glt `He has to fix the door.'  
\z

\noindent The OM \emph{-(w)o} co-references to a 3rd person masculine singular object different from `the door' that didn't appear on the surface. Again, even if the subject of the clause need not be present on the surface (optionally dropped), as a bearer of the obligation, it is implicated on the dyadic verb `fix' and the `door' has not become the theme argument as the {\scshape acc}-marker \emph{nɨ-} on it and the OM tracking it attaches on the verb demonstrate.

Thus, the above Agree-based account of ClP and MoN not only accounts for the similarity between the two but also their important differences in Tigrinya.

\section{Conclusion}\label{sec:gebregziabher:7}

In this paper, I investigated the nature of object markers (OMs) in Tigrinya ClP and MoN and claimed that they exhibit the same characteristics as canonical agreement/subject markers. Consistent with subject markers but unlike clitics, Tigrinya OMs are: (i) not freely movable, (ii) not optional, (iii) formally different from D elements, (v) inadmissible more than one per clause, (vi) insensitive to case alternations, (vii) morphologically idiosyncratic, and (viii) non-referential. Based on these properties, I concluded that Tigrinya OMs are agreement markers that co-vary with the φ-features of their associate DPs. However, it is observed that the OM in Tigrinya is not a 100\% agreement affix type. It exhibits a few properties that appear to be properties of clitics: semantic sensitivity and aspect invariance. These properties, however, are not always helpful to clearly separate clitics from agreement affixes (see \citealt{harizanov14} for discussion).\footnote{Perhaps, these properties hold in Tigrinya due to OMs have not fully finalized their grammaticalization journey from clitics to canonical agreement affix (in the sense of \citealt{fub05}).}

Using a number of morphosyntactic properties, I also argued that the noun phrase the OM tracks is a ``quirky'' subject that re-merges higher to avoid intervention effects, and that \emph{ʔalləw-} is a spell-out of the {\scshape appl} head, which mediates the semantic and syntactic relationship between the two arguments involved in both ClP and MoN.
 If this analysis is on the right track, then it lends support to Agree-based accounts of the object marker (e.g., \citealt{bakerandkramer13}, \citealt{compton14}, a.o) and to alternative accounts to the operation Agree (e.g., \citealt{preminger09}, \citealt{bejarandrezac09}). It also provides empirical support for possessive accounts (e.g., \citet{ritterandrosen1997}, a.o) that posit functional heads as underspecified argument-introducers for ClP, as well as for possessive and MoN accounts, such as \citet{cowperandbjorkman2016} that syntactically the connection between the two.
 
 One of the issues that still remains open is whether all the other OMs in Tigrinya (with ditransitive and unaccusatives) would also receive the same treatment or not. For instance, experiencer and dative arguments (unlike theme or source ones) often require an obligatory OM in Tigrinya. In other Ethio-Semitic languages, such as Amharic, the requirement appears to be optional (\citealt{Leslau:1995}). Future research with a comparative eye will address these issues.

\section*{Acknowledgements}

I would like to thank my Tigrinya informants for sharing their knowledge with me. I would also like to thank two anonymous reviewers and Sharon Rose for their extensive feedback. Thanks also to Elizabeth Cowper for comments and discussion, Betsy Ritter for helpful references, and the Syntax/Semantics Research Group participants at the University of Toronto for discussion on earlier versions of this paper. Thanks also to ACAL\,50 and ADWP\,3 participants for discussion and questions, especially Ken Safir, Mark Baker and Ruth Kramer, for helpful one-on-one discussions; for helping to make the \LaTeX\ format LSP-compatible Mike Diercks and for comments with editing, \LaTeX formatting Felix Kopecky. All errors of course remain solely my own. This research was partly funded by SSHRC postdoctoral fellowship granted to the author.


\section*{Abbreviations}
\begin{multicols}{3}
\begin{tabbing}
{\scshape appl}\hspace{1em}\=Clausal possession\kill
1/2/3 \> persons\\ 
{\scshape acc} \>  accusative\\
{\scshape appl} \> applicative\\ 
ClP \>  clausal possession\\ 
{\scshape d} \> determiner\\ 
{\scshape dat} \> dative\\ 
{\scshape f} \> feminine\\ 
{\scshape ipfv} \> imperfective\\ 
{\scshape m} \> masculine\\ 
MoN \> modal necessity\\ 
{\scshape nom} \> nominative\\ 
{\scshape pl} \> plural\\ 
{\scshape om}  \>  object marker\\ 
{\scshape pfv} \> perfective\\ 
{\scshape o} \> object\\ 
{\scshape s} \> subject\\ 
{\scshape sm} \> subject marker
\end{tabbing}
\end{multicols}

{\sloppy\printbibliography[heading=subbibliography,notkeyword=this]}

\end{document}
