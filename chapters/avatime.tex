\documentclass[output=paper]{langscibook}
\ChapterDOI{10.5281/zenodo.5578818}

%This is where you put the authors and their affiliations
\author{Kerri Devlin\affiliation{UCLA} and Blake Lehman\affiliation{UCLA} and Travis Major\affiliation{UCLA} and Harold Torrence\affiliation{UCLA}}

%Insert your title here
\title{A note on \textit{wh}-questions in Avatime}  
\abstract{This paper examines \textit{wh}-question constructions in Avatime, an endangered Ghana Togo Mountain language.  
We focus on the different wh-strategies and island constraints
 and discuss these in relation to left peripheral elements such as question particles and complementizers.}

\begin{document}
\SetupAffiliations{mark style=none}
\maketitle

\section{Introduction}


In this paper, we present some initial results of an investigation into interrogative constructions in Avatime. We focus on description of the morpho-syntax of \textit{wh}-question formation.

\subsection{Background}
Avatime is an endangered Kwa language spoken in several towns and villages in the Volta Region of Ghana.  More precisely, most Avatime speakers live in an area to the east of Lake Volta approximately 30 miles to the north of Ho, the regional capital. It is also known as \textit{Siya} and \textit{Sideme}.  There are approximately 24,000 native speakers (Ethnologue).

Avatime is classified as a Ghana-Togo Mountain language (also referred to as ``Central Togo” or ``Togo Remnant” languages). The group consists of 15 languages, all of which are spoken in the Volta Region of Ghana and neighboring regions of Togo and Benin.\footnote{Whether the Ghana-Togo Mountain languages constitute a true genetic unit has been debated, with some suggesting instead that they are a geographical, socio- cultural or typological group \citep{ameka2017ghana}.} Within the Ghana-Togo Mountain language group, Avatime belongs to the Ka-Togo branch.

Avatime is a tone language, with three or four level tones \citep{ford:1971, dakubu1988central, schuh1995aspects,defina2009aspect, van2014information}. (In this paper, we assume three tone levels.\footnote{The three levels correspond to Low, High, and Extra High from \citet{ford:1971} and \citet{van2014information}. Unmarked vowels have high tone.  Low tone is indicated by a grave accent and Extra High tone is indicated by an acute accent.})

Avatime has basic SVO word order:

%(1)
\ea \label{ex:devlin:svo}
\gll \`{A}yap\`{e} a-kl\`{a} ke-plekp\`{a}\\
Ayape \textsc{3sg}.\textsc{perf}-read \textsc{cl}-book\\
\glt `Ayape read the book.'\footnote{The orthography used here is based on that used in the Avatime New Testament \citep{gillbt2017}.}
\z

\begin{sloppypar}
Like other GTM languages, Avatime has an active noun class system.  In \REF{ex:devlin:svo}, the singular direct object, \textit{ke-plekp\`{a}} `book’ has a noun class prefix, \textit{ke-}.  Overall, Avatime has 13 noun classes, with the difference between singular and plural typically indicated by a change in the noun class prefix (e.g., \textit{\'{ɔ}/b\'{a}-dzɛ} `woman/women’, \textit{\`{ɔ}/\`{ɪ}-ha} `pig/pigs, \textit{ki/bi-k\`{u}} `yam/yams’, \textit{ku/b\`{e}-d\`{e}} `road/roads).
\end{sloppypar}

\subsection{Interrogatives in Avatime}
Yes/no questions in Avatime can be marked by the presence of a clause-final question particle \textit{na} \citep{ford:1971}:

\ea \label{ex:devlin:na-basic}
\begin{xlist}
\ex 
\gll \'{ɛɛ}-dɔ sras\`{ɛ}\\
\textsc{3sg}.\textsc{prog}-sleep sleep\\
\glt `He is sleeping.’ \hfill \raisebox{1.2\baselineskip}[0pt][0pt]{\textit{Declarative}}
\ex
\gll  \'{ɛɛ}-dɔ sras\`{ɛ} (\textbf{na})?\\
\textsc{3sg}.\textsc{prog}-sleep sleep \textsc{q}\\
\glt `Is he sleeping?’ \hfill \raisebox{1.2\baselineskip}[0pt][0pt]{\textit{Yes/No}}
\end{xlist}
\z


When the overt question particle is absent, a yes/no question may be indicated by the presence of a right-edge falling tone, a characteristic of a number of West African languages \citep{downing2017}.\footnote{Thanks to Laura Downing (p.c.) for pointing this out. However, \citet[62--63]{van2014information} notes, ``Polar questions without final \textit{na} can sometimes be distinguished from statements by intonation as they may end with a high boundary tone. This is only audible when the sentence ends in a low tone and it is optional; there are also cases of polar questions that cannot be distinguished from statements based on intonation." We have found that there are also cases where there is a high right boundary tone that is realized on the final syllable of the constituent that immediately precedes \textit{na}. We leave precise characterization of the tone and intonation of polar questions for future research.}  It is not clear what semantic or pragmatic differences correlate with questions that have the question particle versus those that lack it.  Like yes/no questions, the question particle may optionally appear on the right edge of a \textit{wh}-question:
%(3)
\ea \label{ex:devlin:yesno-basic}
\begin{xlist}
\ex 
\gll w\`o-zulu ke-plekp\`{a} \\
\textsc{2sg.perf}-steal \textsc{cl}-book\\
\glt `You stole the book.’ \hfill \raisebox{1.2\baselineskip}[0pt][0pt]{\textit{Declarative}}
\ex
\gll eg\'{e} w\`{o}-zulu (\textbf{na})?\\
what \textsc{2sg.perf}-steal	 \textsc{q}\\
\glt `What did you steal?’ \hfill \raisebox{1.2\baselineskip}[0pt][0pt]{\textit{wh-Question}}
\end{xlist}
\z

Just as in yes/no questions, if the question particle is absent, there is a falling tone on \textit{ege} `what’ or whatever element occurs on right edge of the clause. (See the examples in (\ref{ex:devlin:what-basic}--\ref{ex:devlin:howmuch-basic}).) In addition, the semantic or pragmatic import of the optional question particle is not clear.  In elicitation, the particle can be used with echo questions.  However, the particle does not, by itself, indicate an echo question given that speakers use it in out-of-the-blue contexts and it can be absent in echo questions.
Anticipating the following section, Avatime \textit{wh}-expressions are given in \tabref{tab:devlin:table1} below.

\begin{table}[ht!]
 \centering
    \begin{tabular}{ll}
    \lsptoprule
\multicolumn{2}{c}{\textit{wh}-expressions}	\\
\midrule
ny\`{a}(ŋ)we, amwe & `who' \\
\`{e}g\`{e} & `what' \\
wol\`{i} & `which' \\
nɪf\`{ɔ} & `where' \\
l\`{i}pol\`{i}, l\'{i}po\`{e}  w\'{o}l\`{i} & `when' \\
kɩtɛ & 'how' \\
ege loso, kɩtɛ loso & `why' \\
k\'{ɪ}t\'{ɛ} t\'{u}(w\'{a})s\`{e} & `how much, how many' \\
\lspbottomrule
\end{tabular}
\caption{\label{tab:devlin:table1}Avatime \textit{wh}-expressions}
\end{table}



Morphologically, \textit{wh}-items appear to be somewhat eclectic. There does not seem to be a truly monomorphemic word corresponding to \textit{when} if \textit{lipoli} `when’ is really a contracted or shortened form of \textit{lipoe} \textit{woli} `when (lit. `which time’).  Similarly, the two expressions for \textit{why} are complex, with one form corresponding to `what reason’ (\textit{ege loso} `why’) and the other corresponding to `how reason’ (\textit{kɩtɛ loso} `why’).  In addition, the word for \textit{where} seems to be built from the general preposition (``P'') \textit{n\'{i}} (\textit{nɩfɔ} `where’).\footnote{If this is on the right track, we still do not know what the \textit{fɔ} corresponds to.}  We also note that we have been unable to detect any distributional difference between the two forms for \textit{who}.  While an individual speaker may have a preference for using one of the forms, all of our consultants accepted all of the forms.

\section{Main clause \textit{wh}-questions}
Avatime main clause \textit{wh}-questions allow for the \textit{wh}-item to appear in the left periphery of the clause and for \textit{wh} in-situ (with or without question particle):

\ea \label{ex:devlin:what-basic}
\begin{xlist}
\ex 
\gll \textbf{eg\'{e}}  wɔ-ŋ\`{a} (na)\\
what \textsc{2sg}.\textsc{perf}-eat \textsc{q}\\
\glt `What did you eat?’ \hfill \raisebox{1.2\baselineskip}[0pt][0pt]{\textit{what} Ex-situ}
\ex
\gll wɔ-ŋ\`{a} \textbf{eg\`{e}}	(na)\\
\textsc{2sg}.\textsc{perf}-eat what \textsc{q}	\\
\glt `What did you eat?’ \hfill \raisebox{1.2\baselineskip}[0pt][0pt]{\textit{what} In-situ}
\end{xlist}
\ex\label{ex:devlin:when-basic}
\begin{xlist}
\ex
\gll \textbf{l\`{i}pol\'{i}} \`{A}yap\`{ɛ} a-yɔ n\'{i} kis\`{a} (na)\\
when Ayape  \textsc{3sg}.\textsc{perf}-wake.up P ? \textsc{q} \\
\glt `When did Ayape wake up?’
\hfill \raisebox{1.2\baselineskip}[0pt][0pt]{\textit{when} Ex-situ}
\ex
\gll \`{A}yap\`{ɛ} a-yɔ n\'{i} kis\`{a} \textbf{l\`{i}pol\`{i}} (na)\\
ayape \textsc{3sg}.\textsc{perf}-wake.up P ? when \textsc{q} \\
\glt `When did Ayape wake up?'
\hfill \raisebox{1.2\baselineskip}[0pt][0pt]{\textit{when} In-situ}
\end{xlist}
\ex \label{ex:devlin:howmuch-basic}
\begin{xlist}
\ex 
\gll \textbf{tu\`{a}s\`{ɛ}} \textbf{ki-mim\'{i}} \'{o}nyime e-dzi (na)  \\
how.much rice man-\textsc{def} \textsc{\textsc{3sg}}.\textsc{perf}-buy \textsc{q}\\
\glt `How much rice did the man buy?’\footnote{Note that in this example the high tone that indicates left peripheral focus is realized on the final vowel of \textit{ki-mimi} `rice', the rightmost word in the fronted constituent.} \hfill \raisebox{1.2\baselineskip}[0pt][0pt]{\textit{how much} Ex-situ}
\ex
\gll \'{o}nyime e-dzi \textbf{tu\`{a}s\`{ɛ}}  \textbf{kimim\`{i}} (na)\\
man-\textsc{def} \textsc{\textsc{3sg}}.\textsc{perf}-buy	how.much rice  \textsc{q} \\
\glt `How much rice did the man buy?’ \hfill \raisebox{1.2\baselineskip}[0pt][0pt]{\textit{how much} In-situ}
\end{xlist}
\z


Avatime does not seem to have a segmental focus marker \citep{ford:1971,van2014information}.  Instead, a constituent (whether a \textit{wh}-expression or not) focused to the left edge of the clause has an extra high tone on its right edge, as can be seen in the examples above. 


While all of the other \textit{wh}-expressions can appear in-situ, \textit{why} patterns differently. \textit{Why} can’t appear in-situ:

%(7)
\ea  \label{ex:devlin:why-basic}
\begin{xlist}
\ex[]{
\gll ege los\'{o} wo-tsyi m\'{a}ŋgo-e (na)\\
what reason \textsc{2sg.perf}-peel  mango-\textsc{def} \textsc{q}\\
\glt `Why did you peel the mango?’ \hfill \raisebox{1.2\baselineskip}[0pt][0pt]{\textit{why} Ex-situ}}
\ex[*]{
\gll wo-tsyi m\'{a}ŋgo-e \textbf{ege} \textbf{los\`{o}} (na)\\
 \textsc{2sg.perf}-peel mango-\textsc{def} what reason \textsc{q}\\
\glt Intended: `Why did you peel the mango?’ \hfill \raisebox{1.2\baselineskip}[0pt][0pt]{\textit{*why} In-situ}}
\end{xlist}
\z

\begin{sloppypar}
Similar asymmetries involving \textit{why} have been documented in other Kwa languages (Krachi: \citealt{torrence2015wh}), distantly related Niger-Congo languages (Bakweri: \citealt{marlo2007exponence}; Zulu: \citealt{buell2011zulu}; Lubukusu: \citealt{wasike2007left}) and in languages unrelated to Avatime (Italian: \citealt{rizzi2001position}; Romanian: \citealt{shlonskysoare}; New Testament Greek: \citealt{kirk2010greek}; Persian: \citealt{karimi2008minimalist}; English: \citealt{hornstein1995logical}; Korean \& Japanese: \citealt{ko2005syntax}; Chinese: \citealt{lin1992syntax}).
\end{sloppypar}



The idea that \textit{why} is syntactically different from the other \textit{wh}-items is supported by the interaction of predicate clefting (and other left peripheral foci) and \textit{why} questions. \REF{ex:devlin:pred-cleft-basic} is an example of a predicate cleft, which involves a nominalized copy of the verb in the left periphery of the clause. \REF{ex:devlin:pred-cleft-what} shows that a predicate cleft is compatible with a \textit{wh}-question, but only when the \textit{wh}-item does not occur on the left edge. (The curly bracketed strings indicate different places where the \textit{wh}-item might appear.) \REF{ex:devlin:pred-cleft-where} shows that an adjunct \textit{wh}-item like \textit{nɩfɔ} `where’ patterns identically to an argument:
%(8)
\ea \label{ex:devlin:pred-cleft}
\begin{xlist}
\ex 
\gll be-dzi ke-plekp\`{a} k\`{e}-dzya m\`{ɛ}\\
\textsc{3pl.perf}-buy \textsc{cl}-book \textsc{cl}-market P\\
\glt `They bought a book at the market.’ 

\ex \label{ex:devlin:pred-cleft-basic}
\gll [\textbf{ki-dz\'{i}}] be-\textbf{dzi} ke-plekp\`{a} k\`{e}-dzya m\`{ɛ}\\
\textsc{nom}-buy \textsc{3pl.perf}-buy \textsc{cl}-book	 \textsc{cl}-market P\\
\glt `They BOUGHT a book at the market (as opposed to, say, selling one).’ \hfill %\raisebox{1.2\baselineskip}[0pt][0pt]\textit{Predicate Cleft}
\ex \label{ex:devlin:pred-cleft-what} %There should be a checkmark next to the third "ege"
\gll \textbf{\{*ege\}} \textbf{[ki-dzi]} \textbf{\{*ege\}} be-dzi \textbf{\{\ding{51}ege\}} k\`{e}-dzya m\`{ɛ}\\
what \textsc{nom}-buy what \textsc{3pl.perf}-buy what \textsc{cl}-market P\\
\glt `What did they BUY  at the market (as opposed to sell there)?’

\ex \label{ex:devlin:pred-cleft-where}%There should be a checkmark next to the third {*nɩfɔ}
\gll \textbf{\{*nɩfɔ\}} \textbf{[ki-dz\`{i}]} \textbf{\{*nɩfɔ\}} be-dzi ke-plekpla \textbf{\{\ding{51}nɩfɔ\}} \\
where \textsc{nom}-buy where \textsc{3pl.perf}-buy \textsc{cl}-book where \\
\glt `Where did they BUY the book (as opposed to sell it)?’
\end{xlist}
\z


This pattern can be made sense of if the clefted predicate and the \textit{wh}-items compete for the same left peripheral focus position. Surprisingly, left peripheral \textit{why} is compatible with a predicate cleft: 


%(9)
\ea \label{ex:devlin:pred-clef-why}
\gll \textbf{\{ege} \textbf{los\'{o}\}} [ki-dzi] \textbf{\{*ege} \textbf{los\'{o}\}} be-dzi ke-plekp\`{a} k\`{e}-dzia m\`{ɛ}\\
what reason \textsc{nom}-buy what reason \textsc{3pl.perf}-buy \textsc{cl}-book \textsc{cl}-market P\\
\glt `Why did they BUY a book at the market (as opposed to, say, selling one)?’
\z


First, these data suggest that \textit{why} occupies a distinct position in the left periphery from other expressions. In addition, the data in \REF{ex:devlin:pred-clef-why} show that \textit{why} must occur to the left of the clefted predicate, which suggests that \textit{why} occupies a position higher than the left peripheral focus position.



\section{Embedded \textit{wh}-questions}
All \textit{wh}-expressions can appear in the left periphery of an embedded question or in situ (except for \textit{why}), with or without a right edge question particle:

%(10)
\ea\label{ex:devlin:embedded}
\begin{xlist}
\ex
\gll me-ʋ\'{i} K\`{o}fi s\`{i} \textbf{\{eg\'{e}\}} be-dzi \textbf{\{ege\}} (na)\\
\textsc{1sg.perf}-ask kofi \textsc{comp} what \textsc{3pl.perf}-buy what Q\\
\glt `I asked Kofi what they bought.'
\ex
\gll me-ʋ\'{i} K\`{o}fi s\`{i} \textbf{\{n\`{ɩ}f\'{ɔ}\}} be-dzi l\`{i}-mwɛ-n\`{ɛ} \textbf{\{n\`{ɩ}f\'{ɔ}\}} (na)\\
\textsc{1sg.perf}-ask kofi \textsc{comp} where \textsc{3pl.perf}-buy \textsc{cl}-orange-\textsc{det}  where Q  \\
\glt `I asked Kofi where they bought the oranges.’
\ex
\gll me-ʋ\'{i} K\`{o}fi s\`{i} \textbf{\{kɩt\'{ɛ}\}} be-dzi l\`{i}-mwɛ-n\`{ɛ} \textbf{\{kɩt\'{ɛ}\}} (na)\\
\textsc{1sg.perf}-ask kofi \textsc{comp} how \textsc{3pl.perf}-buy \textsc{cl}-orange-\textsc{def} how \textsc{Q}  \\
\glt `I asked Kofi how they bought the oranges.’
\end{xlist}
\z


As with matrix questions, it is unclear how the presence or absence of the question particle affects the interpretation of embedded questions. We note, impressionistically, that the question particle is offered much more frequently in elicitation with embedded questions than matrix questions.\footnote{Avatime also allows for embedded yes/no questions that look like their matrix clause counterparts. A reviewer points out that many Kwa languages lack true embedded \textit{wh}-questions and employ a relative clause-like structure instead (see, for example, \citealt{torrence2015wh} on Krachi, a North Guang (Kwa) language of Ghana). Interestingly, Ikpana, a member of the Na-Togo branch within the GTM group, also has true embedded \textit{wh}-questions and a relative clause-like construction, as reported in \citet{ikpana}. At this point, we can say that true embedded \textit{wh}-questions are found in at least one language in both branches of the Ghana-Togo Mountain languages.} 


The embedded questions in \REF{ex:devlin:embedded} are all introduced by the complementizer \textit{sɪ}, which is homophonous with a verb meaning `say’. The \textit{sɪ} complementizer also introduces embedded propositions:
\ea \label{ex:devlin:si-proposition}
\gll  \`{A}yap\`{ɛ} e-b\`{u} s\`{i} M\'{ɛ}r\`{i} e-dzi l\`{i}-mwɛ-n\`{ɛ}\\
ayape \textsc{3sg.perf}-think \textsc{comp} mary \textsc{3sg.perf}-buy \textsc{cl}-orange-\textsc{def}\\
\glt `Ayape thinks that Mary bought oranges.’
\z 
\noindent It is also possible for embedded \textit{wh}-questions to exhibit ``double complementizers”:
%(12)
\ea\label{ex:devlin:double-comp}
\begin{xlist}
\ex \label{ex:devlin:double-comp-a}
\gll me-ʋ\'{i} K\`{o}fi \textbf{s\`{i}} \uline{eg\'{e}} \textbf{s\`{i}} \`{A}yapɛ e-ye\\
\textsc{1sg.perf}-ask Kofi \textsc{comp} what \textsc{comp} ayape \textsc{3sg.perf}-kill \\
\glt `I asked Kofi  what Ayape killed.’\footnote{To our knowledge, the existence of cases like \REF{ex:devlin:double-comp-a} is first noted in \citet[113]{van2014information}.}
\ex
\gll  me-ʋ\'{i} K\`{o}fi \textbf{s\`{i}} \uline{n\`{i}f\'{ɔ}} \textbf{s\`{i}} be-dzi ɔ-k\`{ɔ}-lɔ\\
\textsc{1sg.perf}-ask Kofi \textsc{comp} where \textsc{comp} 	\textsc{3pl.perf}-buy \textsc{cl}-chicken-\textsc{def}  \\
\glt `I asked Kofi where they bought the chicken.’
\end{xlist}
\z
%\footnote{To our knowledge, the existence of cases like \REF{ex:devlin:double-comp-a} is first noted in \cite{van2014information}}.

In \REF{ex:devlin:double-comp}, the \textit{wh}-expression is sandwiched between two instances of the complementizer.  This is immediately reminiscent of the so-called ``CP recursion” structures  found in colloquial English varieties.  \citet{mccloskey2006questions} analyzes English clauses where multiple complementizers appear:


\ea
\begin{xlist}
\ex 
I don’t think that he should contend \uline{\textbf{that} just because he makes}\newline \uline{a promise \textbf{that}} it becomes a responsibility of the United States.\footnote{These are adapted from \citet[(69d) and (69e)]{mccloskey2006questions}.}
\ex 
It is useful to know \uline{\textbf{that} once you have mastered the chosen}\newline \uline{dialect \textbf{that}} you will be able to pick up a news paper and read it.
\end{xlist}
\z

In Avatime, just as in English, it is unclear if there is a semantic or pragmatic difference between the ``simple'' and ``recursive'' CP structures.  Some speakers do report that the multiple COMP construction adds a flavor of emphasis, but this response is inconsistent.  Thus, we leave it here as an open question.

\section{\textit{Wh}-items from embedded clauses}
\textit{Wh}-items associated with argument/adjunct positions in embedded clauses that take matrix scope exhibit three patterns. A \textit{wh}-item from an embedded clause can exhibit full movement and surface on the left edge of the matrix clause \REF{ex:devlin:full-move}. \REF{ex:devlin:part-move} shows that Avatime also allows for partial \textit{wh}-movement, in which the \textit{wh}-item surfaces on the left edge of the embedded clause.  Finally, the \textit{wh}-item in \REF{ex:devlin:insitu} can surface in-situ in its base position in the embedded clause.

%(14)
\ea \label{ex:devlin:embed-obj}\textit{Direct object}
\begin{xlist}
\ex
\gll  K\`ofi e-b\`{u} s\`{i} \`{A}yap\`{ɛ} e-ye ɔ-k\`{ɔ}-lɔ k\'{i} Kwam\`{e}\\
Kofi \textsc{3sg.pres}-think \textsc{comp} Ayape \textsc{3sg.perf}-kill \textsc{cl}-chicken-\textsc{def} \textsc{p} Kwame\\
\glt `Kofi thinks that Ayape killed a chicken for Kwame’

%\hfill \raisebox{1.2\baselineskip}[0pt][0pt]\textit{Direct object}%was above (14b) in original paper
\ex \label{ex:devlin:full-move}
\gll \textbf{eg\'{e}} K\`{o}fi e-b\`{u} s\`{i} \`{A}yapɛ e-ye k\'{i} Kwam\`{e}\\
what Kofi \textsc{3sg.pres}-think \textsc{comp} Ayape \textsc{3sg.perf}-kill \textsc{p}  Kwame \\
\glt `What does Kofi think that Ayape killed for Kwame?’

\ex \label{ex:devlin:part-move}
\gll  K\`{o}fi e-b\`{u} s\`{i} \textbf{eg\'{e}} \`{A}yapɛ e-ye k\'{i} Kwam\`{e}\\
kofi \textsc{3sg.pres}-think \textsc{comp} what Ayape \textsc{3sg.perf}-kill \textsc{p} Kwame \\
\glt `What does Kofi think that Ayape killed for Kwame?’

\ex \label{ex:devlin:insitu}
\gll K\`{o}fi e-b\`{u} s\`{i} \`{A}yapɛ e-ye \textbf{eg\'{e}} k\'{i} Kwam\`{e}\\
kofi \textsc{3sg.pres}-think \textsc{comp} Ayape \textsc{3sg.perf}-kill what \textsc{p} Kwame \\
\glt `What does Kofi think that Ayape killed for Kwame?’
\end{xlist}
\z


\REF{ex:devlin:embed-subj} shows a similar pattern for subjects. The embedded \textit{wh}-subject can surface in its scope position \REF{ex:devlin:embed-subj-scope} or in the embedded clause \REF{ex:devlin:embed-subj-embedded}.  Because a partially moved \textit{wh}-item surfaces to the right of the complementizer, it is not clear whether \REF{ex:devlin:embed-subj-embedded} is a case of partial \textit{wh}-movement or an in-situ \textit{wh}-item. Note too that \REF{ex:devlin:embed-subj-scope} shows that there is no \textit{that-t} effect (as pointed out by a reviewer):

%(15) 
\ea\label{ex:devlin:embed-subj}\textit{Subject}
\begin{xlist} %In original paper it said (15) Subject -- as a label for this list, idk how to format that
\ex \label{ex:devlin:embed-subj-scope}
\gll \textbf{ny\`{a}w\'{ɛ}} K\`{o}fi e-b\`{u} s\`{i} e-ye ɔ-k\`{ɔ}-lɔ k\'{i} Kwam\`{e}\\
who Kofi \textsc{3sg.pres}-think \textsc{comp} \textsc{3sg.perf}-kill \textsc{cl}-chicken-\textsc{def} \textsc{p} Kwame\\
\glt `Who does Kofi think killed the chicken for Kwame?’
\ex \label{ex:devlin:embed-subj-embedded}
\gll K\`{o}fi e-b\`{u} s\`{i} ny\`{a}w\'{ɛ} e-ye ɔ-k\`{ɔ}-lɔ k\'{i} Kwam\`{e}\\
Kofi \textsc{3sg.pres}-think \textsc{comp} who \textsc{3sg.perf}-kill \textsc{cl}-chicken-\textsc{def} \textsc{p} Kwame  \\
\glt `Who does Kofi think killed the chicken for Kwame?’
\end{xlist}
\z

\begin{sloppypar}
The data in \REF{ex:devlin:loc-adj} show that locative adjuncts pattern like arguments. \REF{ex:devlin:loc-adj-full} shows full movement of the adjunct, while \REF{ex:devlin:loc-adj-part} is an example of partial movement. In \REF{ex:devlin:loc-adj-insitu}, the \textit{wh}-item occurs in its postverbal base position.  
\end{sloppypar}

%(16)
\ea \label{ex:devlin:loc-adj}
\begin{xlist}
\ex \label{ex:devlin:loc-adj-decl}
\gll Kof\`{i} e-b\`{u} s\`{i} \`{A}yap\`{ɛ} e-ye ɔ-k\`{ɔ}-l\`{ɔ} k\'{i} Kwam\`{e} n\'{i} k\`{ɛ}-dzia m\`{ɛ}\\ 
Kofi \textsc{3sg.pres}-think \textsc{comp} Ayape \textsc{3sg.perf}-kill \textsc{cl}-chicken-\textsc{def} \textsc{p} Kwame \textsc{loc} \textsc{cl}-market \textsc{p} \\ 
\glt `Kofi thinks that Ayape killed a chicken for Kwame at the market’

\ex \label{ex:devlin:loc-adj-full}
\gll \textbf{nɪf\'{ɔ}} Kof\`{i} e-b\`{u} s\`{i} \`{A}yap\`{ɛ} e-ye ɔ-k\`{ɔ}-l\`{ɔ} k\'{i} Kwam\`{e}\\
where Kofi \textsc{3sg.pres}-think \textsc{comp} Ayape \textsc{3sg.perf}-kill \textsc{cl}-chicken-\textsc{def} \textsc{p} Kwame \\
\glt `Where does Kofi think that Ayape killed a chicken for Kwame?’

\ex \label{ex:devlin:loc-adj-part}
\gll Kof\`{i} e-b\`{u} s\`{i} \textbf{nɪf\'{ɔ}} \`{A}yap\`{ɛ} e-ye ɔ-k\`{ɔ}-l\`{ɔ} k\'{i} Kwam\`{e}\\
Kofi \textsc{3sg.pres}-think \textsc{comp} where Ayape \textsc{3sg.perf}-kill \textsc{cl}-chicken-\textsc{def} \textsc{p} Kwame \\
\glt `Where does Kofi think that Ayape killed a chicken for Kwame?’

\ex \label{ex:devlin:loc-adj-insitu}
\gll Kof\`{i} e-b\`{u} s\`{i} \`{A}yap\`{ɛ} e-ye ɔ-k\`{ɔ}-l\`{ɔ} k\'{i} Kwam\`{e} (n\'{i}) \textbf{nɪf\`{ɔ}}\\
Kofi \textsc{3sg.pres}-think \textsc{comp} Ayape \textsc{3sg.perf}-kill \textsc{cl}-chicken-\textsc{def} \textsc{p} Kwame \textsc{loc} where \\
\glt `Where does Kofi think that Ayape killed a chicken for Kwame?’
\end{xlist}
\z


Note that in the non-interrogative \REF{ex:devlin:loc-adj-decl} the locative PP has both a prepositional element, \textit{n\'{i}} `\textsc{loc}’ and a postposition \textit{mɛ} (see \citealt{ford:1971}). Neither of these occurs with a moved \textit{wh}-item (\ref{ex:devlin:loc-adj-full}, \ref{ex:devlin:loc-adj-part}), although the preposition is optional when the locative \textit{wh} is in-situ.


\section {The relative type of \textit{wh}-interrogation}

The cases examined up to this point have all been ones in which all of the clauses involved have the form of ordinary root clauses.  A second class of \textit{wh}-question strategies involves structures akin to relative clauses (RCs).  To see this, we first consider headed relative clauses:


%(17)
\ea\label{ex:devlin:rc}
\begin{xlist}
\ex \label{ex:devlin:rc-def}
\gll ma-m\`{ɔ} k\`{a}-dru\'{i} \uline{k\`{a}(-kɔ)-l\`{ɔ}} \textbf{g\`{i}/*s\`{ɪ}} wɔ-kp\'{ɔ} *(\`{ɛ})\\
\textsc{1sg.perf}-see \textsc{cl}-dog \textsc{cl}-\textsc{dem}-\textsc{det} \textsc{rel}/*\textsc{comp} \textsc{2sg.perf}-praise \textsc{cdet} \\
\glt `I saw the dog that you praised.'

\ex\label{ex:devlin:rc-indef}
\gll ma-m\`{ɔ} k\`{a}-dru\'{i} \uline{k\`{a}-tɔ} \textbf{g\`{i}/*s\`{ɪ}} wɔ-kp\'{ɔ} *(\`{ɛ}) \\
\textsc{1sg.perf}-see \textsc{cl}-dog \textsc{cl}-\textsc{det} \textsc{rel}/*\textsc{comp} \textsc{2sg.perf}-praise \textsc{cdet}  \\
\glt `I saw a dog that you praised’

\ex\label{ex:devlin:rc-plural}
\gll ma-m\`{ɔ} k\`{u}-dru\'{i} \uline{k\`{u}-tɔ} \textbf{g\`{i}/*s\`{ɪ}} wɔ-kp\'{ɔ} *(\`{ɛ})\\
\textsc{1sg.perf}-see \textsc{cl.pl}-dog \textsc{cl.pl}-\textsc{det} \textsc{rel}/*\textsc{comp} \textsc{2sg.perf}-praise \textsc{cdet}  \\
\glt `I saw some dogs that you praised’
\end{xlist}
\z

Moving from left to right, in Avatime RCs, the left edge of RC (underlined in (\ref{ex:devlin:rc}) has a noun class agreement marker, potentially followed by a demonstrative element (if the head is definite), followed by a determiner.  The underlined string in the definite headed RC in \REF{ex:devlin:demonstrative} indeed occurs independently as a demonstrative:

\ea \label{ex:devlin:demonstrative}
\gll kp\`{ɔ} ka-dru\'{i} \textbf{\uline{k\`{a}-kɔ-l\`{ɔ}}}   \\
praise.\textsc{imper} \textsc{cl}-dog \textsc{cl}-\textsc{dem}-\textsc{def}\\
\glt `Praise that dog!’
\z



The left-edge class markers and determiners are followed by the relative clause complementizer, \textit{gi} (in bold).  This complementizer also occurs on the left edge of a subset of temporal and conditional clauses.  As indicated in \REF{ex:devlin:rc} too, the \textit{sɪ} complementizer, which introduces embedded propositions and questions, cannot be used to introduce a relative clause.  Verbs in Avatime relative clauses do not carry any special morphology and inside of the relative clause, the normal SVO word order is observed. 

The right edge of RCs is indicated by an obligatory \textit{clausal determiner}, \textit{\textsc{cdet}}.  As \citet{van2014information} notes, the clausal determiner assimilates to height and ATR value of a preceding vowel.\footnote{While van Putten notes that the clausal determiner is \textit{often} used, our consultants 
consistently used it in headed relative clauses.}   The clausal determiner is homophonous with one of the definite determiners, for example, the vowel lengthening that distinguishes some definite from indefinite nouns:

%(19)-- maybe formatted wrong
\ea
\begin{xlist}
\ex 
\'{o}-nyime `(a) man'
\ex
\'{o}-nyime\textbf{-\`e} `the man'
\end{xlist}
\z

While it has the form of a definite determiner, the clausal determiner has no obvious relationship to specificity/definiteness, as it occurs with both definite and indefinite headed RCs.  Overall, headed RCs are structured as below:

\ea N  \textsc{cl}-(\textsc{dem})-\textsc{det} [\textsubscript{TP}   SVO ]   \textsc{cdet}
\z

There are two types of relative interrogative structures.  The first involves only the relative marker \textit{gi}:\footnote{At this point, we do not know whether there is an interpretive or pragmatic difference between the relative types of \textit{wh}-question and the non-relative types discussed earlier.  We leave this as a topic for future research.}

\ea
\begin{xlist}
\ex
\gll \textbf{ny\`{a}w\'{ɛ}} \textbf{g\`{i}} a-d\`{a} l\`{i}-mwɛ-n\`{ɛ} (*ɛ)\\
who \textsc{rel} \textsc{2sg.perf}-sell \textsc{cl}-oranges-\textsc{def} \textsc{cdet}  \\
\glt `Who sold the oranges?’\footnote{There are other contexts where \textit{gi} occurs without the final determiner:
\begin{exe}
\sn
\gll  be-b\`{u} s\`{i} g\`{i} wo-dz\`{i} ke-plekp\`{a} \\
3\textsc{pl}-think.\textsc{pres} \textsc{comp} \textsc{rel} 2\textsc{sg}-buy.\textsc{perf} \textsc{cl}-book \\
\glt `They think that you bought a book.’
\end{exe}
}\\

\ex
\gll \textbf{n\`{ɪ}f\'{ɔ}} \textbf{g\`{i}} be-dz\`{i} l\`{i}-mwɛ-n\`{ɛ} (*ɛ)\\
where \textsc{rel} \textsc{3pl.perf}-buy \textsc{cl}-oranges-\textsc{def} \textsc{cdet}  \\
\glt `Where did they buy the oranges?’
\ex
\gll \textbf{eg\'{e}} \textbf{g\`{i}} be-ye (*e) \\
what \textsc{rel} \textsc{3pl.perf}-kill \textsc{cdet}  \\
\glt `What did they kill?'
\end{xlist}
\z

%adding footnote here stops compile, will add later-DONE

\noindent As indicated, unlike ordinary headed RCs, the clause-final determiner is ungrammatical in the \textit{wh}-question construction. 

The RC type of \textit{wh}-interrogative does not allow for \textit{wh} in-situ:


%(22)
\ea
\gll (*g\`{i}) wo-dz\`{i} l\`{i}-mwɛ-n\`{ɛ} \textbf{nɩf\`{ɔ}}\\
 \textsc{rel} \textsc{2sg.perf}-buy \textsc{cl}-oranges-\textsc{def} where\\
\glt `Where did you buy the oranges?’
\z


It is important to note that there is no inherent problem with \textit{g\`{i}} occurring by itself on the left edge of a clause.  This configuration can be found in some temporal clauses, with the right edge clausal determiner:
%(23)
\ea 
\gll mɛ-t\'{a}-dɔ sras\`{ɛ} [g\`{i} K\`{o}fi e-dz\`{i} \`{i}] \\
\textsc{1sg}-\textsc{fut}-sleep sleep \textsc{rel} Kofi \textsc{3sg}-arrive \textsc{cdet}\\
\glt `I will sleep when Kofi arrives.’\footnote{Some temporal/conditional clauses may also involve the relative complementizer and \textit{whether}: 
\begin{exe}
\sn
\gll ma\'{a}-dɔ sras\`{ɛ} [x\'{ɛ} g\`{i} Kofi o-dz\`{i} \`{i}]  \\
\textsc{1sg.fut}-sleep sleep whether \textsc{rel} kofi \textsc{3sg}-arrive \textsc{cdet}\\
\glt `I will sleep when Kofi arrives.’
\end{exe}
\noindent It is not clear whether the clausal determiner always appears in conditional or temporal clauses.
}
\z

%add footnote

The second type of RC-related \textit{wh}-interrogative clause involves the RC complementizer and the noun class agreeing markers. In \REF{ex:devlin:nc-dem-det}, the left-edge \textit{wh}-expression is immediately followed by the string \textit{class marker}\,+\,(\textit{demonstrative})\,+ \textit{determiner}.  This complex is followed by the relative marker \textit{gi}. As before, the clausal determiner is not permitted.  \REF{ex:devlin:nc-dem-det-no-insitu} shows that the \textit{wh}-expression cannot be in-situ in this construction. (\ref{ex:devlin:nc-dem-det-which}--\ref{ex:devlin:nc-dem-det-embed}) show this type of relative clause \textit{wh}-construction for other \textit{wh}-expressions in matrix and from embedded clauses:


%(24)
\ea \label{ex:devlin:rc-noun-class}
\begin{xlist}
\ex \label{ex:devlin:nc-dem-det}
\gll \textbf{eg\'{e}} ka(-kɔ)-l\`{ɔ} g\`{i} be-dzi (*\`{i})\\
what \textsc{cl}-\textsc{dem}-\textsc{det} \textsc{rel} \textsc{3pl.perf}-buy \textsc{cdet}  \\
\glt `What did they buy?’

\ex \label{ex:devlin:nc-dem-det-no-insitu}
\gll *ka(-kɔ)-l\`{ɔ} g\`{i} be-dzi (*\`{i}) \textbf{eg\`{e}} \\
\textsc{cl}-\textsc{dem}-\textsc{det} \textsc{rel} \textsc{3pl.perf}-buy \textsc{cdet} what   \\
\glt `What did they buy?’

\ex \label{ex:devlin:nc-dem-det-which}
\gll \textbf{bi-k\`{u}} \textbf{w\`{o}l\'{i}} \textbf{bɛ-(wɔ)-l\`{ɔ}} \textbf{g\`{i}} be-dzi (*\`{i}) \\
\textsc{cl.pl}-yam which \textsc{cl}-\textsc{dem}-\textsc{det} \textsc{rel} \textsc{3pl.perf}-buy \textsc{cdet} \\
\glt `Which yams did they buy?’

\ex \label{ex:devlin:nc-dem-det-where}
\gll \textbf{nɪf\'{ɔ}} \textbf{l\'{ɛ}-(wɔ)-l\`{ɔ}} \textbf{g\`{i}} be-dzi l\`{i}-mwɛ-n\`{ɛ} (*\`{ɛ})\\
where \textsc{cl}-\textsc{dem}-\textsc{det} \textsc{rel} \textsc{3pl.perf}-buy \textsc{cl.pl}-oranges-\textsc{def} \textsc{cdet}\\
\glt `Where did they buy the oranges?’

\ex \label{ex:devlin:nc-dem-det-embed}
\gll \textbf{bi-k\`{u}} \textbf{w\`{o}l\'{i}} \textbf{bɛ-l\`{ɔ}} \textbf{g\`{i}} be-b\`{u} s\`{i} be-dzi (*\`{i})\\
\textsc{cl.pl}-yam which \textsc{cl}-\textsc{det} \textsc{rel} \textsc{3pl.pres}-think \textsc{comp} 3\textsc{pl.perf}-buy.\textsc{perf} \textsc{cdet}\\
\glt `Which yams do they think that they bought?’
\end{xlist}
\z

It is also possible to form complex chains with multiple relative markers.  \REF{ex:devlin:mult-rel-base} gives the base case, with a \textit{wh}-phrase on the left edge. \REF{ex:devlin:mult-rel-rel} shows the relative \textit{wh}-construction.  In \REF{ex:devlin:mult-rel-kakalo}, the higher clause looks the same as in \REF{ex:devlin:mult-rel-rel}, but the left edge of the embedded clause is marked by the presence of noun-class agreeing pronominal (\textit{ka}) and a \textit{class marker}\,+\,\textit{determiner} string, which looks very similar to what occurs on the left edge of a headed relative clause.  In \REF{ex:devlin:mult-rel-simple-matrix}, the matrix clause has none of the left edge material found in relative clauses, but embedded clause has the pronominal \textit{ka} and the \textit{class marker}\,+\,\textit{determiner} string found in \REF{ex:devlin:mult-rel-kakalo}.
%(25)
\ea \label{ex:devlin:mult-rel}
\begin{xlist}
\ex \label{ex:devlin:mult-rel-base}
\gll \textbf{ka-dru\`{i}} \textbf{w\`{o}l\'{i}} be-b\`{u} s\`{i} \`{A}yapɛ a-kp\`{ɔ} (*ɛ)\\
\textsc{cl}-dog which \textsc{3pl.pres}-think \textsc{comp} Ayape 3\textsc{sg.perf}-praise \textsc{cdet}  \\
\glt `Which dog do they think that Ayape praised?’

\ex \label{ex:devlin:mult-rel-rel}
\gll \textbf{ka-dru\`{i}} \textbf{w\`{o}l\'{i}} \textbf{ka-l\`{ɔ}} \textbf{g\`{i}} be-b\`{u} s\`{i} \`{A}yapɛ a-kp\`{ɔ} (*ɛ)\\
\textsc{cl}-dog which \textsc{cl}-\textsc{det} \textsc{rel} \textsc{3pl.pres}-think \textsc{comp} Ayape 3\textsc{sg.perf}-praise \textsc{cdet}  \\
\glt `Which dog do they think that Ayape praised?’

\ex \label{ex:devlin:mult-rel-kakalo}
\gll \textbf{ka-dru\`{i}} \textbf{w\`{o}l\'{i}} \textbf{ka-l\`{ɔ}} \textbf{g\`{i}} be-b\`{u} s\`{i} \uline{\textbf{ka\'{a}-ka-l\`{ɔ}}} \%(g\`{i}) \`{A}yapɛ a-kp\`{ɔ} (*ɛ) \\
\textsc{cl}-dog which \textsc{cl}-\textsc{det} \textsc{rel} 3\textsc{pl.pres}-think \textsc{comp} \textsc{cl}-\textsc{cl}-\textsc{det} \textsc{rel} Ayape 3\textsc{sg.perf}-praise \textsc{cdet}%\footnote{The "\%" used here indicates that some speakers consistently judged the presence of the relative marker as grammatical, while other speakers consistently rejected sentences where the relative marker was present in this position.}
\\
\glt `Which dog do they think that Ayape praised?’\footnote{The ``\%'' used here and in (\ref{ex:devlin:mult-rel-simple-matrix})  indicates that some speakers consistently judged the presence of the relative marker as grammatical, while other speakers consistently rejected sentences where the relative marker was present in this position.}

\ex \label{ex:devlin:mult-rel-simple-matrix}
\gll \textbf{ka-dru\`{i}} \textbf{w\`{o}l\'{i}} be-b\`{u} s\`{i} \uline{\textbf{ka\'{a}-ka-l\`{ɔ}}} \%(g\`{i}) \`{A}yap\`{ɛ} a-kp\`{ɔ} (*ɛ)\\
\textsc{cl}-dog which 3\textsc{pl.pres}-think \textsc{comp} \textsc{cl}-\textsc{cl-det} \textsc{rel} Ayape 3\textsc{sg.perf}-praise \textsc{cdet}  \\
\glt `Which dog do they think that Ayape praised?’
\end{xlist}
\z

Note that the class-agreeing element \textit{ka} in (\ref{ex:devlin:mult-rel-rel}--\ref{ex:devlin:mult-rel-simple-matrix}) can also appear in simple \textit{wh}-questions, but only in pre-subject position:
%(26)
\ea
\begin{xlist}
\ex %a
\gll *ka-dru\`{i} w\`{o}l\'{i} wɔ-m\`{ɔ} \textbf{k\'{a}}\\
\textsc{cl}-dog which 2\textsc{sg.perf}-see \textsc{cl} \\
\glt `Which dog did you see?’

\ex %b
\gll ka-dru\`{i} w\`{o}l\'{i} \textbf{k\'{a}} wɔ-m\`{ɔ}\\
\textsc{cl}-dog which \textsc{cl} 2\textsc{sg.perf}-see \\
\glt `Which dog did you see?’
\end{xlist}
\z


\section{Islands}
Turning to islands \citep{ross1967constraints}, there cannot be an island boundary between the surface position of a \textit{wh}-item and its base position. We first consider the complex noun phrase constraint (CNPC),  cases like relative clauses, such as \REF{ex:devlin:cnpc-base}.  A case like \REF{ex:devlin:cnpc-move} represents an attempt to move the \textit{wh}-item out of the RC, yielding ungrammaticality. Note too that the presence of a resumptive pronoun (\textit{wa})  does not alleviate the ungrammaticality:


%(27)
\ea \label{ex:devlin:cnpc}
\begin{xlist}
\ex \label{ex:devlin:cnpc-base}
\gll bɛ-kp\`{ɔ} \'{o}-nyime lɛ-kɔ-l\`{ɔ} [g\`{i} a-d\`{a} ki-mim\`{i}-ɛ ɛ]\\
3\textsc{pl.perf}-praise \textsc{cl}-man \textsc{cl-?-det} \textsc{rel} 3\textsc{sg.perf}-sell \textsc{cl}-rice-\textsc{def} \textsc{cdet}\\
\glt `They praised the man who sold the rice.’

\ex \label{ex:devlin:cnpc-move}
\gll *\textbf{eg\'{e}} bɛ-kp\`{ɔ} \'{o}-nyime lɛ-kɔ-l\`{ɔ} [g\`{i} a-d\`{a} $\emptyset$/\textbf{wa} e]\\
what 3\textsc{pl.perf}-praise \textsc{cl}-man	 \textsc{cl}-?-\textsc{det} \textsc{rel} 3\textsc{sg.perf}-sell $\emptyset$/\textsc{cl.pron} \textsc{cdet}\\
\glt 
\end{xlist}
\z

However, if the \textit{wh}-item is left inside of the island, the result is grammatical:

%(28)
\ea
\gll bɛ-kp\`{ɔ} ó-nyime lɛ-kɔ-l\`{ɔ} [g\`{i} a-d\`{a} \textbf{ ege} e] \\
3\textsc{pl.perf}-praise \textsc{cl}-man \textsc{cl}-?-\textsc{det} \textsc{rel} 3\textsc{sg.perf}-sell what \textsc{cdet} \\
\glt `What thing is such that they praised the man who sold it?’
\z

A similar pattern holds for adjunct clauses.  In \REF{ex:devlin:adj-island-base}, the (bracketed) \textit{before} clause has two elements, \textit{xɛ} ‘whether’ and  \textit{able} ‘before’, on the left edge.   The ungrammatical \REF{ex:devlin:adj-island-move} results from attempting to move the \textit{wh}-item out of the adjunct island. Note that the presence of a resumptive pronoun, \textit{y\`{ɛ}} '\textsc{cl.1sg.pron}' or a gap has no effect on the ungrammaticality.  \REF{ex:devlin:adj-island-insitu} shows that if the \textit{wh}-item is left inside of the island, the result is fine.  


%(29)
\ea \label{ex:devlin:adj-island}
\begin{xlist}
\ex \label{ex:devlin:adj-island-base}
\gll  be-dz\`{i}  [x\'{ɛ} \'{a}bl\'{e} \`{A}yap\`{ɛ} a-kp\`{ɔ} K\`{o}fi]\\
3\textsc{pl.perf}-arrive whether before Ayape 3\textsc{sg.perf}-praise Kofi  \\
\glt `They arrived before Ayape praised Kofi.’

\ex \label{ex:devlin:adj-island-move}
\gll  \textbf{*ny\`{a}w\'{e}} be-dz\`{i}  [x\'{ɛ} \'{a}bl\'{e} \`{A}yap\`{ɛ} a-kp\`{ɔ} yɛ/$\_\_$ ]\\
who 3\textsc{pl.perf}-arrive whether before Ayape 3\textsc{sg.perf}-praise \textsc{cl1.sg.pron} {}\\
\glt 

\ex \label{ex:devlin:adj-island-insitu}
\gll be-dz\`{i}  [x\'{ɛ} \'{a}bl\'{e} \`{A}yap\`{ɛ} a-kp\`{ɔ} ny\`{a}w\`{e}]\\
3\textsc{sg.perf}-arrive whether before Ayape 3\textsc{sg.perf}-praise who  \\
\glt `They arrived before Ayape praised who?’ 
\end{xlist}
\z

We next look at coordinate structures like \REF{ex:devlin:csc-base}, where objects are conjoined. \REF{ex:devlin:csc-move-sec-conj}, \REF{ex:devlin:csc-move-first-conj}, both ungrammatical attempts to move a  single conjunct out, immediately suggest that the coordinate structure constraint is active in Avatime.  These examples also show that the presence of a class-agreeing resumptive pronoun or a gap both result in ungrammaticality:

%(30)
\ea \label{ex:devlin:csc}
\begin{xlist}
\ex \label{ex:devlin:csc-base}
\gll  bɛ-d\`{a} [ke-plekp\`{a} n\`{i} \`{ɔ}-mwɛ-n\`{ɔ}]\\
3\textsc{pl.perf}-sell \textsc{cl}-book and \textsc{cl}-orange-\textsc{def}  \\
\glt `They sold a book and an orange.’
%$\emptyset$
%$\_\_$
\ex \label{ex:devlin:csc-move-sec-conj}
\gll \textbf{*\`{e}g\'{e}} bɛ-d\`{a} [ke-plekp\`{a} n\`{i} lɔ/$\emptyset$ ] \\
 what 3\textsc{pl.perf}-sell \textsc{cl}-book and \textsc{cl2.sg.pron} {}\\
\glt `What did they sell the book  and?’\footnote{This sentence can  only have the meaning `By using what did they sell the book?' Also, the pronoun \textit{lɔ} is drawn from noun class 2, hence the gloss.}

\ex \label{ex:devlin:csc-move-first-conj}
\gll \textbf{*\`{e}g\'{e}} bɛ-d\`{a} [ka/$\emptyset$ n\`{i} \`{ɔ}-mwɛ-n\`{ɔ}]\\
what 3\textsc{pl.perf}-sell \textsc{cl6.sg.pron}{} and \textsc{cl}-orange-\textsc{def} \\
\glt `What did they sell and the orange?’\footnote{This can only have the meaning `What did they sell by means of/using an orange?'}
\end{xlist}
\z

As with the other island cases, if the \textit{wh}-item is in-situ in either conjunct, the result is grammatical:

%(31)
\ea
\begin{xlist}
\ex %a
\gll bɛ-d\`{a} [ke-plekp\`{a} n\`{i} \textbf{eg\`{e}}]\\
3\textsc{pl.perf}-sell \textsc{cl}-book and what  \\
\glt `They sold a book and what?’

\ex %b
\gll  bɛ-d\`{a} [\textbf{\`{e}ge} n\`{i} \`{ɔ}-mwɛ-n\`{ɔ}]\\
3\textsc{pl.perf}-sell what and \textsc{cl}-orange-\textsc{def}  \\
\glt `They sold what and an orange?’
\end{xlist}
\z


The cases that we have looked at so far, RCs, adjuncts, and conjoined structures are known to be strong islands. The second case covered by the CNPC is CP complements to nouns. While degraded in English, argument extractions from CP complements to nouns is permitted (`\textsuperscript{?/??}What did you hear the rumor that John stole at the store?’). On the other hand, adjunct extraction is ungrammatical (`*Where\textsubscript{k} did you hear the rumor that John stole the book \textit{t\textsubscript{k}}?', in which, informally, the `where' is construed in the embedded clause and the request is for the location of the stealing event.) In Avatime, in contrast, apparent CP complements to nouns are transparent for extraction for both arguments \REF{ex:devlin:cp-comp-move} and at least some adjuncts \REF{ex:devlin:cp-comp-move-when}.  However, unlike English, these do not look like relative clauses because there is no relative complementizer (\textit{gi}) or clause-final determiner.

%(32)
\ea \label{ex:devlin:cp-comp}
\begin{xlist}
\ex \label{ex:devlin:cp-comp-base}
\gll me-n\'{u} liwɔlɛm\`{ɛ} s\`{i} \`{A}yapɛ a-d\`{a} ke-plekp\`{a} n\'{i} k\`{ɛ}-dzia m\`{ɛ}\\
1\textsc{sg.perf}-hear rumor \textsc{comp} Ayape 3\textsc{sg.perf}-sell \textsc{cl}-book \textsc{loc} \textsc{cl}-market P \\
\glt `I heard the rumor that Ayape sold a book at the market.’

\ex \label{ex:devlin:cp-comp-move}
\gll \`{e}g\'{e} wo-n\'{u} liwɔlɛm\`{ɛ} s\`{i} \`{A}yapɛ a-d\`{a} n\'{i} k\`{ɛ}-dzia m\`{ɛ}\\
what 2\textsc{sg.perf}-hear rumor \textsc{comp} Ayape 3\textsc{sg.perf}-sell \textsc{loc} \textsc{cl}-market P  \\
\glt `What did you hear the rumor that Ayape sold at the market?’

\ex \label{ex:devlin:cp-comp-move-when}
\gll \textbf{nɪf\'{ɔ}} wo-n\'{u} liwɔlɛm\`{ɛ} s\`{i} \`{A}yapɛ a-d\`{a} ke-plekp\`{a}\\
where 2\textsc{sg.perf}-hear rumor \textsc{comp} Ayape 3\textsc{sg.perf}-sell \textsc{cl}-book\\
\glt `Where did you hear the rumor that Ayape sold a book?’
\end{xlist}
\z

While cases like \REF{ex:devlin:cp-comp-move} and \REF{ex:devlin:cp-comp-move-when} are the translational equivalents to the English, it is very likely that they are not direct structural analogues of the  English cases. Note that the noun \textit{liwɔlɛm\`{ɛ}} `rumor’ is followed by the \textit{s\`ɪ} complementizer, which is banned in headed relative clauses.  Further, as the examples show, both arguments and adjuncts can be A-bar extracted, which is unlike what is found in English.\footnote{Thanks to a reviewer for suggesting this discussion.} 


\section{Summary}
In the preceding sections, we have reviewed \textit{wh}-question strategies in Avatime and some of their syntactic and morphological properties.  Like other Kwa languages such as Krachi \citep{torrence2015wh}, Avatime allows for \textit{wh}-movement, \textit{wh}-in-situ, and for partial \textit{wh}-movement.  The Avatime data are particularly interesting because they highlight the cross-linguistic question of exactly what semantic or pragmatic differences (if any) are encoded in different morpho-syntactic \textit{wh}-strategies, a topic that must be left to future investigation.  Zooming out, we have also shown that, while Avatime does exhibit island sensitivity, the in-situ strategy allows formation of genuine questions from inside islands.  This calls for further investigation of the parameters of variation in these effects crosslinguistically. 

\section*{Acknowledgments}
The research is supported by a grant from the National Science Foundation (BCS-1748590). We thank our Avatime native speaker consultants: Gifty Amu, Peace Awunyama, Vincent Azafokpe, Wisdom Ekissi, Jones Kwame, Philomena Kumatse, Paul Kwawu, Akos Mawulorm and Agbenya Wisdom. Many thanks to Dr. Kofi Dorvlo, who was critical to the success of the project. We greatly appreciate the comments and suggestions from two anonymous reviewers and community proofreaders.


{\sloppy\printbibliography[heading=subbibliography,notkeyword=this]}
\end{document}
